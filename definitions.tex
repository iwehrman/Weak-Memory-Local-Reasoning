%!TEX root = dissertation.tex

\usepackage{stackrel}

%% types

\newcommand{\setlocations}{\ensuremath{\mathbb{L}}}
\newcommand{\setvalues}{\ensuremath{\mathbb{V}}}
\newcommand{\setintegers}{\ensuremath{\mathbb{Z}}}
\newcommand{\setnaturals}{\ensuremath{\mathbb{N}}}
\newcommand{\setpositives}{\ensuremath{\mathbb{N}^+}}
\newcommand{\setidentifiers}{\ensuremath{\mathbb{I}}}
\newcommand{\setbooleans}{\ensuremath{\mathbb{B}}}
\newcommand{\setprocessors}{\ensuremath{\mathbb{P}}}

\newcommand{\setevents}{\ensuremath{\mathbb{E}}}
\newcommand{\proc}[1]{\ensuremath{\mathit{proc}(#1)}}
\newcommand{\eqproc}{\ensuremath{\approx}}
\newcommand{\tp}[1]{\ensuremath{\mathit{type}(#1)}}
\newcommand{\tpflush}{\ensuremath{\mathtt{flush}}}
\newcommand{\tpwrite}[1]{\ensuremath{\mathtt{write}(#1)}}
\newcommand{\tpalloc}[1]{\ensuremath{\mathtt{alloc}(#1)}}
\newcommand{\writesto}[1]{\ensuremath{\mathit{Wr}_{#1}}}
\newcommand{\allocs}[1]{\ensuremath{\mathit{Al}_{#1}}}
\newcommand{\writes}{\ensuremath{\mathit{Wr}}}
\newcommand{\flushes}{\ensuremath{\mathit{Fl}}}


%% abbreviations
\newcommand{\setstores}{\ensuremath{\mathbf{Store}}}
\newcommand{\setstacks}{\ensuremath{\mathbf{Stack}}}
\newcommand{\setheaps}{\ensuremath{\mathbf{Heap}}}
\newcommand{\setbuffers}{\ensuremath{\mathbf{Buf}}}
\newcommand{\setshares}{\ensuremath{\mathbf{Share}}}
\newcommand{\setperms}{\ensuremath{\mathbf{Perm}}}
\newcommand{\setastates}{\ensuremath{\mathit{\Sigma}}}
\newcommand{\setstates}{\ensuremath{\mathbf{State}}}
\newcommand{\setpstates}{\ensuremath{\Pi}}
\newcommand{\setmpstates}{\ensuremath{\mathbf{MPState}}}
\newcommand{\setresolvers}{\ensuremath{\mathbf{Resolver}}}
\newcommand{\setconfigurations}{\ensuremath{\mathbf{Config}}}
\newcommand{\setmconfigurations}{\ensuremath{\mathbf{MConfig}}}
\newcommand{\setassertions}{\ensuremath{\mathbf{Assert}}}

%% operators 

\DeclareMathOperator{\op}{\mathit{op}}
\DeclareMathOperator{\opskip}{\mathsf{skip}}
\DeclareMathOperator{\opfence}{\mathsf{fence}}
\DeclareMathOperator{\opassume}{\mathsf{assume}}
\DeclareMathOperator{\opassert}{\mathsf{assert}}
\DeclareMathOperator{\opassign}{:\!=}
\DeclareMathOperator{\opequals}{=}
\DeclareMathOperator{\opseq}{;}
\DeclareMathOperator{\opchoice}{+}
\DeclareMathOperator{\oppar}{||}
\DeclareMathOperator{\oploop}{\ast}
\DeclareMathOperator{\oplock}{\mathsf{lock}}
\DeclareMathOperator{\opunlock}{\mathsf{unlock}}
\DeclareMathOperator{\opcons}{\mathsf{cons}}
\DeclareMathOperator{\opnew}{\mathsf{new}}
\DeclareMathOperator{\opalloc}{\mathsf{alloc}}
\DeclareMathOperator{\opdispose}{\mathsf{free}}
\DeclareMathOperator{\opif}{\mathsf{if\,}}
\DeclareMathOperator{\opthen}{\mathsf{\,then\,}}
\DeclareMathOperator{\opelse}{\mathsf{\,else}}
\DeclareMathOperator{\opwhile}{\mathsf{while\,}}
\DeclareMathOperator{\opdo}{\mathsf{\,do\,}}
\DeclareMathOperator{\oplocal}{\mathsf{local\,}}
\DeclareMathOperator{\opin}{\mathsf{\,in\,}}
\DeclareMathOperator{\opresource}{\mathsf{resource\,}}
\DeclareMathOperator{\opwith}{\mathsf{with\,}}
\DeclareMathOperator{\opwhen}{\mathsf{\,when\,}}
\DeclareMathOperator{\nil}{\emptyset}
\DeclareMathOperator{\compat}{\smile}



%% primitive commands

\newcommand{\cskip}[0]{\ensuremath{\opskip}}
\newcommand{\cfence}[0]{\ensuremath{\opfence}}
\newcommand{\clock}[0]{\ensuremath{\oplock}}
\newcommand{\cunlock}[0]{\ensuremath{\opunlock}}
\newcommand{\cassume}[1]{\ensuremath{\opassume(#1)}}
\newcommand{\cassert}[1]{\ensuremath{\opassert(#1)}}
\newcommand{\cassign}[2]{\ensuremath{#1 \opassign #2}}
\newcommand{\cload}[2]{\ensuremath{#1 \opassign \left[ #2 \right]}}
\newcommand{\cstore}[2]{\ensuremath{\left[ #1 \right] \opassign #2}}
\newcommand{\cnew}[2]{\ensuremath{#1 \opassign \opnew(#2)}}
\newcommand{\calloc}[1]{\ensuremath{#1 \opassign \opalloc}}
\newcommand{\ccons}[3]{\ensuremath{#1 \opassign \opcons(#2, #3)}}
\newcommand{\cfree}[1]{\ensuremath{\opdispose(#1)}}

%% commands

\newcommand{\cseq}[2]{\ensuremath{#1 \opseq #2}}
\newcommand{\cchoice}[2]{\ensuremath{#1 \opchoice #2}}
\newcommand{\cpar}[2]{\ensuremath{#1 \oppar #2}}
\newcommand{\cifthenelse}[3]{\ensuremath{\opif #1 \opthen #2 \opelse #3}}
\newcommand{\cifthen}[2]{\ensuremath{\opif #1 \opthen #2}}
\newcommand{\cwhile}[2]{\ensuremath{\opwhile #1 \opdo #2}}
\newcommand{\cloop}[1]{\ensuremath{#1^{\oploop}}}
\newcommand{\clocal}[3]{\ensuremath{\oplocal #1 \opassign #2 \opin #3}}
\newcommand{\cres}[2]{\ensuremath{\opresource #1 \opin #2}}
\newcommand{\cwith}[2]{\ensuremath{\opwith #1 \opdo #2}}
\newcommand{\catomic}[1]{\ensuremath{\left\langle #1 \right\rangle}}
\newcommand{\ccas}[1]{\ensuremath{\mathsf{cas}(#1)}}

\newcommand{\exprs}{\ensuremath{\mathbf{Expr}}}
\newcommand{\fracs}{\ensuremath{\mathbf{Frac}}}
\newcommand{\bexprs}{\ensuremath{\mathbf{BExpr}}}
\newcommand{\pcomms}{\ensuremath{\mathbf{PComm}}}
\newcommand{\comms}{\ensuremath{\mathbf{Comm}}}
\newcommand{\asserts}{\ensuremath{\mathbf{Assert}}}
\newcommand{\progs}{\ensuremath{\mathbf{Prog}}}

\newcommand{\dnexpr}[1]{\ensuremath{\mathcal{E}[\![#1]\!]}}
\newcommand{\dnshare}[1]{\ensuremath{\mathcal{S}[\![#1]\!]}}
\newcommand{\dnbexpr}[1]{\ensuremath{\mathcal{B}[\![#1]\!]}}
\newcommand{\dnpcomm}[1]{\ensuremath{\mathcal{P}[\![#1]\!]}}
\newcommand{\dna}[1]{\ensuremath{\mathcal{A}[\![#1]\!]}}
\newcommand{\dnfrac}[1]{\ensuremath{\mathcal{F}[\![#1]\!]}}

\newcommand{\sflush}[1]{\ensuremath{\mathit{flush}\left(#1\right)}}
\newcommand{\speek}[2]{\ensuremath{\mathit{peek}_{#1}(#2)}}
\newcommand{\sval}[1]{\ensuremath{\mathit{val}(#1)}}
\newcommand{\spush}[3]{\ensuremath{\mathit{push}_{#1}^{#2}(#3)}}
\newcommand{\alloc}[1]{\ensuremath{\mathit{alloc}(#1)}}
\newcommand{\allocat}[2]{\ensuremath{\mathit{alloc}_{#1}(#2)}}
\newcommand{\owned}[1]{\ensuremath{\mathit{owned}(#1)}}
\newcommand{\full}[1]{\ensuremath{\mathit{full}(#1)}}

\newcommand{\pcstep}[1]{\,\stackrel[#1]{}{\,\hookrightarrow}\,}
\DeclareMathOperator{\step}{\hookrightarrow}
\DeclareMathOperator{\estep}{\rightarrow}
\DeclareMathOperator{\pstep}{\twoheadrightarrow}
\DeclareMathOperator{\noestep}{\nrightarrow}

\newcommand{\taustep}{\ensuremath{\stackrel[\tau]{}{\step}}}
\newcommand{\rttaustep}{\ensuremath{\stackrel[\tau]{\ast}{\step}}}
\newcommand{\exptaustep}[1]{\ensuremath{\stackrel[\tau]{#1}{\step}}}

\newcommand{\poflushat}[1]{\ensuremath{\leq_{#1}}}
\newcommand{\poflush}{\ensuremath{\leq}}

\DeclareMathOperator{\shagree}{\sim}
\DeclareMathOperator{\mpagree}{\propto}
\DeclareMathOperator{\mpcons}{\#}
\DeclareMathOperator{\stcong}{\,\equiv\,}
\newcommand{\fv}[1]{\ensuremath{\mathrm{fv}(#1)}}
\renewcommand{\mod}[1]{\ensuremath{\mathrm{mod}(#1)}}

\DeclareMathOperator{\sep}{\ast}
\DeclareMathOperator{\seq}{;}
\DeclareMathOperator{\emp}{\mathbf{emp}}
\DeclareMathOperator{\barr}{\mathbf{bar}}
\DeclareMathOperator{\pointsto}{\mapsto}
\DeclareMathOperator{\expop}{\mathbf{exp}}
\newcommand{\expand}[1]{\ensuremath{\expop(#1)}}
\newcommand{\sexpand}[1]{\ensuremath{\mathit{expand}(#1)}}

\newcommand{\concst}[1]{\ensuremath{\left\lfloor#1\right\rfloor}}
\newcommand{\abst}[1]{\ensuremath{\left\lceil#1\right\rceil}}
\newcommand{\fullst}[1]{\ensuremath{\mathsf{full}(#1)}}
\newcommand{\writeto}[2]{\ensuremath{\stackrel[#1]{#2}{\leadsto}}}

\newcommand{\kernel}{\ensuremath{\mathcal{K}}}
\DeclareMathOperator{\trim}{\mathit{trim}}
\DeclareMathOperator{\shadd}{\oplus}
\DeclareMathOperator{\shmult}{\odot}

\newcommand{\bexpt}{\ensuremath{\mathbf{true}}}
\newcommand{\bexpf}{\ensuremath{\mathbf{false}}}

\newcommand{\spec}[4]{\ensuremath{#1\;\vdash\;\set{#2}\;#3\;\set{#4}}}
\newcommand{\truespec}[4]{\ensuremath{#1\;\models\;\set{#2}\;#3\;\set{#4}}}
\newcommand{\andif}{\ensuremath{\text{\;\;\;and\;\;\;}}}

\newcommand{\safe}[2]{\ensuremath{\mathsf{safe}_{#1}(#2)}}

\newcommand{\defined}[1]{\ensuremath{\mathsf{def}(#1)}}

\renewcommand{\stcong}[1]{\ensuremath{\sim_{#1}}}
\newcommand{\abort}{\ensuremath{\zorch}}

\newcommand{\downpstates}{\ensuremath{\down \powerset{\setpstates}}}
\newcommand{\pjoin}{\ensuremath{\sqcup}}
\newcommand{\pmeet}{\ensuremath{\sqcap}}
\newcommand{\live}[1]{\ensuremath{\mathit{live}(#1)}}

\newcommand{\femp}{\ensuremath{\mathbf{emp}}}
\newcommand{\fbar}[1]{\ensuremath{\mathbf{bar}(#1)}}
\newcommand{\flock}[1]{\ensuremath{\mathbf{lock}(#1)}}
\newcommand{\fwrite}[1]{\ensuremath{\leadsto_{#1}}}
\newcommand{\fwrote}{\ensuremath{\rightarrowtail}}
\newcommand{\fpointsto}{\ensuremath{\mapsto}}
\newcommand{\fiter}[1]{\ensuremath{#1^{\ast}}}
\DeclareMathOperator{\fhash}{\#}
\DeclareMathOperator{\fsep}{\ast}
\DeclareMathOperator{\fseq}{;}
\DeclareMathOperator{\fsseq}{,}

\newcommand{\presat}[1]{\ensuremath{\kernel(#1)}}
\newcommand{\pred}[1]{\ensuremath{[\![#1]\!]}}
\newcommand{\locked}[1]{\ensuremath{\mathit{locked}(#1)}}

\newcommand{\taurefines}{\ensuremath{\preceq}}