\documentclass[11pt]{article}

\usepackage[bookmarks=true,bookmarksopen=true]{hyperref}
\usepackage{url}
\usepackage{datetime}
\usepackage{proof}
\usepackage{enumerate}
\usepackage{graphicx}

\synctex=1

%!TEX root = dissertation.tex

\usepackage{amsmath,amssymb,amsthm}
\usepackage{bcprules}
\usepackage{graphicx}
\usepackage{stmaryrd}
\usepackage{nicefrac}

\theoremstyle{plain}
\newtheorem{lemma}{Lemma}
\newtheorem{theorem}{Theorem}
\newtheorem{corollary}{Corollary}
\newtheorem{proposition}{Proposition}
\newtheorem{conjecture}{Conjecture}

\theoremstyle{definition}
\newtheorem{definition}{Definition}

\theoremstyle{remark}
\newtheorem{note}{Note}
\newtheorem*{remark}{Remark}


\newcommand{\bnfbar}{\ensuremath{~|~}}
\newcommand{\bnfdef}{\ensuremath{~::=~}}

%% set notation
\newcommand{\set}[1]{\left\lbrace #1 \right\rbrace}
\newcommand{\setof}[2]{\set{#1 \mid #2}}
\newcommand{\bigset}[1]{\left\lbrace #1 \right\rbrace}
\newcommand{\bigsetof}[2]{\bigset{#1 \mid #2}}

\DeclareMathOperator{\funcompat}{\smile}

\newcommand{\mtab}{\ensuremath{\;\;\;}}

\newcommand{\powerset}[1]{\ensuremath{\mathcal{P}(#1)}}
\newcommand{\zorch}{\ensuremath{\lightning}}

\DeclareMathOperator{\opdom}{\mathrm{dom}}
\newcommand{\dom}[1]{\ensuremath{\opdom(#1)}}

\newcommand{\bvt}{\ensuremath{\mathbf{t}}}
\newcommand{\bvf}{\ensuremath{\mathbf{f}}}
\newcommand{\llen}[1]{\ensuremath{|#1|}}
\newcommand{\llit}[1]{\ensuremath{\left[#1\right]}}
\newcommand{\lnil}[0]{\ensuremath{\varepsilon}}
\newcommand{\elems}[1]{\ensuremath{|\!|#1|\!|}}
\newcommand{\funof}[1]{\ensuremath{\overline{#1}}}
\DeclareMathOperator{\lapp}{+\!\!\!+}
\DeclareMathOperator{\lcons}{:\!:}
\newcommand{\lhead}[1]{\ensuremath{\mathit{hd}(#1)}}
\newcommand{\ltail}[1]{\ensuremath{\mathit{tl}(#1)}}
\newcommand{\nth}[2]{\ensuremath{#2 ! #1}}
\newcommand{\listof}[1]{\ensuremath{#1~\mathsf{list}}}

\newcommand{\relset}[1]{\ensuremath{|#1|}}
\newcommand{\ext}[1]{\ensuremath{\hat{#1}}}
\newcommand{\restrict}[2]{\ensuremath{#1 |_{#2}}}
\newcommand{\proj}[2]{\ensuremath{#1 |_{#2}}}

\newcommand{\reclit}[1]{\ensuremath{\set{#1}}}

\DeclareMathOperator{\tfun}{\rightarrow}
\DeclareMathOperator{\pfun}{\rightharpoonup}
\DeclareMathOperator{\fpfun}{\pfun_{\mathrm{fin}}}

\DeclareMathOperator{\st}{\,:\,}

\newcommand{\ptup}[2]{\ensuremath{#1 \shortleftarrow #2}}
\newcommand{\funup}[2]{\ensuremath{#1 \left[ #2 \right]}}
\newcommand{\fundel}[2]{\ensuremath{#1 \setminus#2 }}
\newcommand{\recup}[3]{\funup{#1}{\ptup{#2}{#3}}}
\newcommand{\ptapp}[2]{\ensuremath{#1 \lapp #2}}

\DeclareMathOperator{\conj}{\,\wedge\,}
\DeclareMathOperator{\disj}{\,\vee\,}
\DeclareMathOperator{\onlyif}{\,\Rightarrow\,}
%\DeclareMathOperator{\ifandonlyif}{\,\Longleftrightarrow\,}
\renewcommand{\iff}{\ensuremath{\Leftrightarrow}}

\newcommand{\fresh}[1]{\ensuremath{\mathrm{fresh}(#1)}}
\newcommand{\subst}[2]{\ensuremath{\left[#1 / #2\right]}}
\newcommand{\unsubst}[2]{\ensuremath{\left[#1 \backslash #2\right]}}

\newcommand{\eqclass}[1]{\ensuremath{\left[ #1 \right]}}
\DeclareMathOperator{\downop}{\downarrow}
\newcommand{\downo}[2]{\ensuremath{\downop_{#1}\!#2}}
\newcommand{\down}[1]{\ensuremath{\downop #1}}

%% Calculational proofs

\newcommand{\Calc}[1]{\begin{description}
                \item \begin{tabbing}\qquad\=\quad\=\kill
                #1\end{tabbing}
                %\> \> #1\end{tabbing}
                \end{description}}
\newcommand{\aconn}[1]{\> \{#1\}\\ }
\newcommand{\aconni}[1]{\> \> \{#1\}\\ }
\newcommand{\acomm}[1]{\> \> #1 \\}
\newcommand{\conn}[2]{\\*$#1 $\> \{#2\}\\\> \>}
\newcommand{\Conn}[1]{\conn{=}{#1}}

\newcommand{\ident}[1]{\ensuremath{\mathit{Id}_{#1}}}
\DeclareMathOperator{\rcomp}{\circ}
\newcommand{\relexp}[2]{\ensuremath{\stackrel{#2}{#1}}}
\newcommand{\tcl}[1]{\stackrel{+}{#1}}
\newcommand{\rtcl}[1]{\stackrel{\ast}{#1}}

\newcommand{\image}[2]{\ensuremath{#1\left[#2\right]}}
\newcommand{\invimage}[2]{\ensuremath{#1^{-1}\left[#2\right]}}

\newcommand{\lift}[1]{\ensuremath{\mathit{lift}_{#1}}}
\renewcommand{\merge}{\ensuremath{\uplus}}

% \newcommand{\override}{\reflectbox{\rotatebox[origin=c]{180}{\ensuremath{\,\oslash\,}}}}
\newcommand{\override}{\ensuremath{\backslash\!\!\backslash}}

\newcommand{\sentails}{\ensuremath{\,\models\,}}
\newcommand{\satisfies}{\ensuremath{\,\models\,}}
\newcommand{\sequiv}{\ensuremath{\,\equiv\,}}

\newcommand{\fixme}[1]{\todo{FIXME: #1}}

\DeclareMathOperator{\eqdef}{\;=_{\mathit{df}}\;}
\DeclareMathOperator{\iffdef}{\;\equiv_{\mathit{df}}\;}

\newcommand{\card}[1]{\left|#1\right|}
\newcommand{\cmpl}[1]{\ensuremath{\overline{#1}}}

\newcommand{\refines}{\ensuremath{\sqsubseteq}}
\newcommand{\range}[1]{\ensuremath{\mathit{range}(#1)}}
%!TEX root = dissertation.tex

\usepackage{stackrel}

%% types

\newcommand{\setlocations}{\ensuremath{\mathbb{L}}}
\newcommand{\setvalues}{\ensuremath{\mathbb{V}}}
\newcommand{\setintegers}{\ensuremath{\mathbb{Z}}}
\newcommand{\setnaturals}{\ensuremath{\mathbb{N}}}
\newcommand{\setpositives}{\ensuremath{\mathbb{N}^+}}
\newcommand{\setidentifiers}{\ensuremath{\mathbb{I}}}
\newcommand{\setbooleans}{\ensuremath{\mathbb{B}}}
\newcommand{\setprocessors}{\ensuremath{\mathbb{P}}}

%% abbreviations

\newcommand{\setstacks}{\ensuremath{\mathbf{Stack}}}
\newcommand{\setheaps}{\ensuremath{\mathbf{Heap}}}
\newcommand{\setbuffers}{\ensuremath{\mathbf{Buf}}}
\newcommand{\setshares}{\ensuremath{\mathbf{Share}}}
\newcommand{\setperms}{\ensuremath{\mathbf{Perm}}}
\newcommand{\setastates}{\ensuremath{\mathit{\Sigma}}}
\newcommand{\setstates}{\ensuremath{\mathbf{State}}}
\newcommand{\setmemorysystems}{\ensuremath{\mathbf{Mem}}}
\newcommand{\setpstates}{\ensuremath{\Pi}}
\newcommand{\setmpstates}{\ensuremath{\mathbf{MPState}}}
\newcommand{\setresolvers}{\ensuremath{\mathbf{Resolver}}}
\newcommand{\setconfigurations}{\ensuremath{\mathbf{Config}}}
\newcommand{\setmconfigurations}{\ensuremath{\mathbf{MConfig}}}
\newcommand{\setassertions}{\ensuremath{\mathbf{Asrt}}}

%% operators 

\DeclareMathOperator{\op}{\mathit{op}}
\DeclareMathOperator{\opskip}{\mathsf{skip}}
\DeclareMathOperator{\opfence}{\mathsf{fence}}
\DeclareMathOperator{\opassume}{\mathsf{assume}}
\DeclareMathOperator{\opassert}{\mathsf{assert}}
\DeclareMathOperator{\opassign}{:\!=}
\DeclareMathOperator{\opequals}{=}
\DeclareMathOperator{\opseq}{;}
\DeclareMathOperator{\opchoice}{+}
\DeclareMathOperator{\oppar}{|\!|}
\DeclareMathOperator{\oploop}{\ast}
\DeclareMathOperator{\oplock}{\mathsf{lock}}
\DeclareMathOperator{\opunlock}{\mathsf{unlock}}
\DeclareMathOperator{\opcons}{\mathsf{cons}}
\DeclareMathOperator{\opnew}{\mathsf{new}}
\DeclareMathOperator{\opalloc}{\mathsf{alloc}}
\DeclareMathOperator{\opdispose}{\mathsf{free}}
\DeclareMathOperator{\opif}{\mathsf{if\,}}
\DeclareMathOperator{\opthen}{\mathsf{\,then\,}}
\DeclareMathOperator{\opelse}{\mathsf{\,else}}
\DeclareMathOperator{\opwhile}{\mathsf{while\,}}
\DeclareMathOperator{\opdo}{\mathsf{\,do\,}}
\DeclareMathOperator{\oplocal}{\mathsf{local\,}}
\DeclareMathOperator{\opin}{\mathsf{\,in\,}}
\DeclareMathOperator{\opresource}{\mathsf{resource\,}}
\DeclareMathOperator{\opwith}{\mathsf{with\,}}
\DeclareMathOperator{\opwhen}{\mathsf{\,when\,}}
\DeclareMathOperator{\nil}{\emptyset}
\DeclareMathOperator{\compat}{\smile}


%% primitive commands

\newcommand{\cskip}[0]{\ensuremath{\opskip}}
\newcommand{\cfence}[0]{\ensuremath{\opfence}}
\newcommand{\clock}[0]{\ensuremath{\oplock}}
\newcommand{\cunlock}[0]{\ensuremath{\opunlock}}
\newcommand{\cassume}[1]{\ensuremath{\opassume(#1)}}
\newcommand{\cassert}[1]{\ensuremath{\opassert(#1)}}
\newcommand{\cassign}[2]{\ensuremath{#1 \opassign #2}}
\newcommand{\cload}[2]{\ensuremath{#1 \opassign \left[ #2 \right]}}
\newcommand{\cstore}[2]{\ensuremath{\left[ #1 \right] \opassign #2}}
\newcommand{\cnew}[2]{\ensuremath{#1 \opassign \opnew(#2)}}
\newcommand{\calloc}[1]{\ensuremath{#1 \opassign \opalloc}}
\newcommand{\ccons}[3]{\ensuremath{#1 \opassign \opcons(#2, #3)}}
\newcommand{\cfree}[1]{\ensuremath{\opdispose(#1)}}

%% commands

\newcommand{\cseq}[2]{\ensuremath{#1 \opseq #2}}
\newcommand{\cchoice}[2]{\ensuremath{#1 \opchoice #2}}
\newcommand{\cpar}[2]{\ensuremath{#1 \oppar #2}}
\newcommand{\cifthenelse}[3]{\ensuremath{\opif #1 \opthen #2 \opelse #3}}
\newcommand{\cifthen}[2]{\ensuremath{\opif #1 \opthen #2}}
\newcommand{\cwhile}[2]{\ensuremath{\opwhile #1 \opdo #2}}
\newcommand{\cloop}[1]{\ensuremath{#1^{\oploop}}}
\newcommand{\clocal}[3]{\ensuremath{\oplocal #1 \opassign #2 \opin #3}}
\newcommand{\cres}[2]{\ensuremath{\opresource #1 \opin #2}}
\newcommand{\cwith}[2]{\ensuremath{\left\langle #2\right\rangle_{#1}}}
% \newcommand{\cwith}[2]{\ensuremath{\opwith #1 \opdo #2}}
\newcommand{\catomic}[1]{\ensuremath{\left\langle #1 \right\rangle}}
\newcommand{\ccas}[1]{\ensuremath{\mathsf{cas}(#1)}}

\newcommand{\exprs}{\ensuremath{\mathbf{Expr}}}
\newcommand{\fracs}{\ensuremath{\mathbf{Frac}}}
\newcommand{\bexprs}{\ensuremath{\mathbf{BExpr}}}
\newcommand{\pcomms}{\ensuremath{\mathbf{PComm}}}
\newcommand{\comms}{\ensuremath{\mathbf{Comm}}}
\newcommand{\asserts}{\ensuremath{\mathbf{Assert}}}
\newcommand{\progs}{\ensuremath{\mathbf{Prog}}}

\newcommand{\dnexpr}[1]{\ensuremath{\mathcal{E}[\![#1]\!]}}
\newcommand{\dnshare}[1]{\ensuremath{\mathcal{S}[\![#1]\!]}}
\newcommand{\dnbexpr}[1]{\ensuremath{\mathcal{B}[\![#1]\!]}}
\newcommand{\dnpcomm}[1]{\ensuremath{\mathcal{P}[\![#1]\!]}}
\newcommand{\dna}[1]{\ensuremath{\mathcal{A}[\![#1]\!]}}
\newcommand{\dnfrac}[1]{\ensuremath{\mathcal{F}[\![#1]\!]}}

\newcommand{\sflush}[1]{\ensuremath{\mathit{flush}\left(#1\right)}}
\newcommand{\speek}[2]{\ensuremath{\mathit{peek}_{#1}(#2)}}
\newcommand{\sval}[1]{\ensuremath{\mathit{val}(#1)}}
\newcommand{\spush}[3]{\ensuremath{\mathit{push}_{#1}^{#2}(#3)}}
\newcommand{\alloc}[1]{\ensuremath{\mathit{alloc}(#1)}}
\newcommand{\allocat}[2]{\ensuremath{\mathit{alloc}_{#1}(#2)}}
\newcommand{\owned}[1]{\ensuremath{\mathit{owned}(#1)}}
\newcommand{\full}[1]{\ensuremath{\mathit{full}(#1)}}

\newcommand{\step}{\ensuremath{\rightarrow}}
\newcommand{\rtstep}{\ensuremath{\stackrel{\ast}{\rightarrow}}}
\newcommand{\nostep}{\ensuremath{\nrightarrow}}
\newcommand{\pstep}[1]{\ensuremath{\stackrel{#1}{\rightarrow}}}
\newcommand{\taustep}{\pstep{\tau}}

\newcommand{\fv}[1]{\ensuremath{\mathrm{fv}(#1)}}
\renewcommand{\mod}[1]{\ensuremath{\mathrm{mod}(#1)}}

\newcommand{\bexpt}{\ensuremath{\mathbf{true}}}
\newcommand{\bexpf}{\ensuremath{\mathbf{false}}}


\newcommand{\spec}[4]{\ensuremath{#1\,\vdash\,\set{#2}\;#3\;\set{#4}}}
\newcommand{\truespec}[4]{\ensuremath{#1\,\models\,\set{#2}\;#3\;\set{#4}}}
\newcommand{\proofof}[1]{\ensuremath{\overline{#1}}}
\newcommand{\andif}{\ensuremath{\text{\;\;\;and\;\;\;}}}
\newcommand{\triple}[3]{\ensuremath{\vdash\,\set{#1}\;#2\;\set{#3}}}
\newcommand{\truetriple}[3]{\ensuremath{\models\,\set{#1}\;#2\;\set{#3}}}

\newcommand{\safe}[2]{\ensuremath{\mathsf{safe}_{#1}(#2)}}

\newcommand{\defined}[1]{\ensuremath{\mathsf{def}(#1)}}

\newcommand{\abort}{\ensuremath{\zorch}}

\newcommand{\live}[1]{\ensuremath{\mathit{live}(#1)}}

\newcommand{\femp}{\ensuremath{\mathbf{emp}}}
\newcommand{\fbar}[1]{\ensuremath{\mathbf{bar}_{#1}}}
\newcommand{\flock}[1]{\ensuremath{\mathbf{lock}_{#1}}}
\newcommand{\fexp}[1]{\ensuremath{\mathbf{exp}({#1})}}
\newcommand{\ftop}[1]{\ensuremath{\mathbf{top}_{#1}}}
%\newcommand{\fwrite}[1]{\ensuremath{\!\stackrel{#1}{\leadsto}\!}}
\newcommand{\fwrite}[1]{\ensuremath{\!\leadsto_{#1}\!}}
\newcommand{\fwrote}[1]{\ensuremath{\rightarrowtail_{#1}}}
\newcommand{\fpointsto}{\ensuremath{\!\mapsto\!}}
\newcommand{\fiter}[1]{\ensuremath{#1^{\ast}}}
\DeclareMathOperator{\fhash}{\,\#\,}
\DeclareMathOperator{\fsep}{\,\ast\,}
\DeclareMathOperator{\fseq}{\,\vartriangleleft\,}
\DeclareMathOperator{\fsseq}{\,\blacktriangleleft\,}
\DeclareMathOperator{\fseqc}{\,\vartriangleright\,}
\DeclareMathOperator{\fsseqc}{\,\blacktriangleright\,}
\DeclareMathOperator{\fsepi}{\,\relbar\!\!\ast\,}
\DeclareMathOperator{\fseqli}{\,\relbar\!\vartriangleleft\,}
\DeclareMathOperator{\fseqri}{\,\vartriangleleft\!\relbar\,}
\DeclareMathOperator{\fsepic}{\,\ast\!\!\relbar\,}
\DeclareMathOperator{\fseqlic}{\,\vartriangleright\!\relbar\,}
\DeclareMathOperator{\fseqric}{\,\relbar\!\vartriangleright\,}



\newcommand{\fbaru}{\ensuremath{\mathbf{bar}}}
\newcommand{\fwriteu}{\ensuremath{\!\leadsto\!}}

\newcommand{\pred}[1]{\ensuremath{[\![#1]\!]}}
\newcommand{\locked}[1]{\ensuremath{\mathit{locked}(#1)}}
\newcommand{\taurefines}{\ensuremath{\preceq}}
\newcommand{\complete}[1]{\ensuremath{\mathit{complete}(#1)}}
\newcommand{\stable}[1]{\ensuremath{\mathit{stable}(#1)}}
\newcommand{\stcong}[1]{\ensuremath{\sim_{#1}}}

\setcounter{secnumdepth}{3}
\setcounter{tocdepth}{3}

\begin{document} 

	\settimeformat{ampmtime}
	\author{Ian Wehrman} 
	\title{Weak-Memory Local Reasoning\\Dissertation Draft\footnote{This is a draft. Please do not distribute without permission.}} 
	\date{\today, \currenttime}
	\maketitle

\section*{Preface}

This goal of this document is to bring you, the reader, up-to-date on the status of my dissertation project. I am interested in doing so because I wish to convince you that this project, while far from complete, constitutes sufficient evidence of my doctoral training. 

Surely the most persuasive way for me to convince you of the completeness of my training would be to present a finished project, complete with formal definitions, lemmas and corresponding proofs. Unfortunately, having worked toward this goal for so long, I no longer expect to reach this status in a reasonable amount of time. It may happen tomorrow, or in another year, or perhaps never. I have many times felt that the end was in sight, only to discover some as-yet-unconsidered aspect of the problem to be inconsistent with my earlier efforts and assumptions, sending me back to reconsider the entirety of the development. 

Nonetheless, I have made progress on this problem of which I am genuinely proud. In some instances that progress can be explained  mathematically; elsewhere only informal or intuitive justification can be given for the technical decisions I have made. The reason for the incompleteness of the project after so much time has, I believe, as more to do with the size of the problem than its inherent difficulty. (To that end, see Figure~\ref{fig:dependency-graph}.) Indeed, I am more confident than ever that the approach I have pursued to the problem of local reasoning about weak-memory programs can be brought to a satisfying and enlightening conclusion. But, because of the many components of the project that must stand together in relative harmony, I simply do not know how much longer it will take me to arrive at such a conclusion. 

With those caveats in mind, I do feel ``close'' to having the definitions needed for an program logic that I can prove sound. This logic may be weaker than I had hoped, but I have yet to carefully explore its capabilities and limitations. (It is hard to do this, after all, when the syntax, semantics and proof theory of the logic itself is in flux, as it essentially has been since inception.) Whether or not that is the case, I hope you will agree after reading this document that significant progress has been made toward that milestone, and that the contributions I have made are indicative of satisfactory training.

\section{Project Overview}

The goal of this project is to develop a program logic for local reasoning about structured concurrent programs that have a semantics consistent with the x86-TSO memory model as defined by Owens, Sarkar and Sewell \cite{DBLP:conf/tphol/OwensSS09}. 

The various components of this project and their explicit dependencies are pictured in the (transitively reduced) graph of Figure~\ref{fig:dependency-graph}. The components are represented by shapes that indicate their approximate type: semantic objects by octagons; formal languages by squares; semantic relationships by ovals; deduction systems by hexagons; and key lemmas by trapezoids. 

\begin{figure}[ht]
\begin{center}
\includegraphics[scale=0.5]{dependency-graph/dg-reduced}
\caption{\label{fig:dependency-graph}Dependency graph for the project.}
\end{center}
\end{figure}

The memory model is shown in a double-lined octagon, which reflects the assumption in this project that it is complete and correct, and is not further modified from its operational definition in \cite{DBLP:conf/tphol/OwensSS09}, which is summarized in Section~\ref{sec:memory-model}. 

The machine model---in particular, the notion of machine state---depends on, but is distinct from, the memory model. For example, the memory model dictates that each processor has a private set of named registers, whereas in the machine models I shall define, a single share set of variable names is assumed. I will also take the liberty in the machine model to relax other restrictions on the notion of state from the memory model, such as the requirement that writes buffered by a single processor are totally ordered. 

The programming language is a structured C-like language with concurrent composition. It does not explicitly depend on any other components of the project. The programming language semantics relates the programming language to the machine model, and hence depends on them both. The machine models, programming language and semantics are described in Section~\ref{sec:programming-language}. 

The assertion language also does not explicitly depend on any other components of the project.\footnote{Implicitly, of course, it depends significantly on the memory and machine models.} The assertion language semantics associates the assertion language to predicates, which are defined later as sets of states (as defined by the machine model) with a particular structure. Assertions and their meaning are described in Section~\ref{sec:assertions}. 

Ideally there would also be a proof theory of assertions and a corresponding soundness theorem. I have chosen not to focus on a proof theory of assertions in this project, but will indicate some semantic implications and equivalences that would be relevant to that end in Section~\ref{sec:algebra}.

The specification language encompasses the programming and assertions languages, and its semantics is given in terms of the semantics of programs and assertions. The proof theory of specifications relies on the existence of a suitable proof theory of assertions for determining entailments. The soundness of the specification logic relies on the the soundness of the proof theory of assertions as well as the semantics of programs and assertions. Specifications are described in Section~\ref{sec:specifications}. 

\subsection{Motivation}

Why bother building a program logic? The original motivation was as follows. Although program logics are reasonable systems in which to construct hand proofs of arbitrary program properties, they have more recently been shown to be amenable to automation of relatively shallow properties, e.g., memory safety or shape properties. Unfortunately, existing logics can not be soundly applied to certain fine-grained concurrent programs like concurrent data structures. This is because these programs are typically not well locked and contain races, and so cannot rely on the underlying computer architecture to ensure that their interaction with memory is sequentially consistent. But sequential consistency is a deep assumption in most existing program logics, hence their inapplicability. 

As further motivation, concurrent data structures are inarguably important to computer science given the decline of single-threaded processor performance improvements and concomitant proliferation of parallelism. At the same time, correctness arguments for concurrent data structures are subtle enough to make informal reasoning extremely difficult. Additionally, these programs are of only modest size, which (perhaps) gives cause for optimism about their amenability to automated or semi-automated verification. Altogether, this appears to be an excellent opportunity for the application of formal methods. 

\paragraph{A Reappraisal}

I am less confident in the immediate practical value of this project than I originally was, having identified a number of errors of judgment in the original motivation. First, I was wrong to consider the small size of these programs as increasing the viability of automated reasoning about them. This is backwards: because these subtle and important programs are so small, it is entirely practical to consider hand-constructed formal proofs of their correctness using proof assistants like Coq, Isabelle or ACL2. And although constructing these proofs is difficult, surely it is less so than developing a general technique for doing so. 

Second, although such programs are clearly racy, it is not clear that their interactions with memory fall outside the bounds of sequentially consistency. And for sequentially consistent programs, it seems unlikely that an approach of such high fidelity w.r.t.~the memory model (e.g., with explicit write buffers) will turn out to be the most effective. 

Nonetheless, I still consider the project to have scientific merit. It faces the problem of local reasoning about the behavior of programs executing on a more complicated machine quite directly and gives some indication of how this can be done without relying on simplifying assumptions about memory. Local reasoning techniques can, of course, be quite useful even for hand-constructed formal proofs. 

In the best case, this work could someday provide a foundation for practically useful reasoning about a class of difficult programs. In the worst case, it sheds some light on the the problem of local program reasoning in general by providing an additional---fairly extreme---data point in the space of program logics, illustrating the difficulty and complexity of reasoning about the behavior of programs w.r.t.~a widespread and weak memory model. 

\section{Program Logics}

TODO 

\paragraph{Hoare Logic}

\paragraph{Pointers} Separation Logic and Bunched Implications 

\paragraph{Concurrency} Owicki-Gries; Rely-Guarantee

\paragraph{Pointers and Concurrency} Concurrent Separation Logic; RG-Sep

\section{Memory Consistency Models}

TODO 

\paragraph{Sequential Consistency} 

\subsection{The x86-TSO Memory Model}
\label{sec:memory-model}

This section contains a brief, informal review of the x86-TSO memory model as defined by Owens, Sarkar and Sewell \cite{DBLP:conf/tphol/OwensSS09}. The model is given as a collection of legal traces of memory events. Legal traces are defined axiomatically and operationally, the latter via a labeled transition relation between machine states. These states are four-tuples $(R,m,B,l)$ in which: \begin{itemize}
	\item $R : \setprocessors \tfun \setidentifiers \pfun \setvalues$ is a register file for each processor;
	\item $m : \setlocations \pfun \setvalues$ is a shared memory; 
	\item $B : \setprocessors \tfun (\setlocations \times \setvalues)~\mathsf{list}$ is a write buffer for each processor; 
	\item $l : \setprocessors^\bot$ is a global lock. 
\end{itemize}

Transitions (labeled by memory events) between states indicate the possibility and effect of those memory events. For example, in any state $(R,m,B,l)$, processor $p$ may write a value $v$ into its register $i$; i.e., it may update the state such that $R_p(i) = v$. Similarly, if $R_p(i) = v$ then $p$ may read value $v$ from its register $i$. A summary of the other events processor $p$ may perform is as follows: \begin{itemize}
	\item it may load from its write buffer the most recent value of a location---or, if a write to that location is not found in its write buffer, from memory---if the lock is either available ($i.e.$, the lock value is $\bot$) or is held by $p$, but not if some other processor $q \neq p$ holds the lock;
	\item it may store a value to a memory location by adding a new write to the head of its write buffer regardless of the status of the lock; 
	\item it may flush (or, synonymously in this document, commit) the least recent write in its buffer to memory if it holds the lock or the lock is available;  
	\item it may fence if its buffer is empty;
	\item it may acquire the lock (i.e., change the lock value in the current state to $p$) if the lock is available; 
	\item it may release the lock (i.e., change the lock value in the current state to $\bot$) if it holds the lock.
\end{itemize}

This specification bounds the sort of memory events that can occur in program executions, but it does not give meaning to the programs of a particular language. The semantics of the programming language used in this project is however guided by the memory model, and it ought be possible to prove that it respects the bounds of the model. But this is not the focus of this project; and, even without such a correspondence proof, it will still be clear that the semantics of the programming language is manifestly weak. 


\section{Related Work}

Technical inspiration for this project comes primarily from work on separation logic \cite{DBLP:conf/lics/Reynolds02,DBLP:conf/csl/OHearnRY01,DBLP:journals/bsl/OHearnP99} and abstract separation logic \cite{DBLP:conf/lics/CalcagnoOY07}, as well as concurrent separation logic  \cite{DBLP:journals/tcs/OHearn07,DBLP:journals/tcs/Brookes07}, which this program logic resembles insofar as it strives to enable local (instead of global) reasoning about shared-state invariants (instead of two-place state relations). Although perhaps not obviously so, the semantics of the programming language was influenced by work on graphical models \cite{DBLP:journals/ipl/WehrmanHO09,DBLP:conf/RelMiCS/HoareMSW09,DBLP:journals/jlp/HoareMSW11} and the pomset model of true concurrency from Pratt \cite{DBLP:conf/popl/Pratt82,DBLP:conf/concur/Pratt84}. The style of semantics of specifications, and the associated soundness proof, is taken almost directly from Vafeiadis' recent soundness proof of concurrent separation logic \cite{V11}. Vafeiadis' excellent dissertation has also been an invaluable guide \cite{VafeiadisDissertation}. 

Also of note are two works that present solutions to the same weak-memory reasoning problem, both developed much more fully than the work I shall describe. First is Ridge's rely-guarantee program logic for the x86-TSO memory model \cite{DBLP:conf/vstte/Ridge10}. Ridge's logic is formalized in HOL and has been demonstrated with proofs of a number of interesting algorithms, including Simpsons's 4-slot non-blocking buffer. In contrast to my project, Ridge's is a logic for the x86 assembly language, whereas I target a higher level, structured language. Additionally, Ridge's logic is not inherently local, and offers nothing like the frame rule of separation logic. 

A second work of note is Cohen and Schirmer's \cite{DBLP:conf/itp/CohenS10} reduction from x86-TSO to sequential consistency for certain programs. This is notable because the class of programs they consider is larger than just the well locked programs. They show that many concurrent programming paradigms, although racy, in fact remain sequentially consistent. They furthermore provide a method of syntactic restriction for an Owicki-Gries-style program logic that allows sound reasoning about such programs. Although they describe some useful programs that fall outside of this boundary, this seems to be a work of great practical importance. Although their logic also offers no frame rule, Cohen has suggested in private communication that a similar restriction may be applied concurrent separation logic for sound local reasoning.

Related but less relevant to the current problem, compared to the previous two papers, is work by Ferreira, Feng and Shao which gives soundness proofs for concurrent separation logic in a variety relaxed-memory settings \cite{DBLP:conf/esop/FerreiraFS10}. As with the original soundness proof by Brooks \cite{DBLP:journals/tcs/Brookes07}, their theorem applies to well locked programs only, which are necessarily sequentially consistent. 

\subsection{Algebraic Models of Concurrency}

Graphical models \cite{DBLP:journals/ipl/WehrmanHO09} Concurrent Kleene Algebra \cite{DBLP:conf/RelMiCS/HoareMSW09,DBLP:conf/concur/HoareMSW09} Locality bimonoids \cite{DBLP:conf/concur/HoareHMOPS11} 

The programming language is local w.r.t.\ all the separating conjunctions defined; locality w.r.t.\ sequential composition doesn't seem to be important in the work above.  

\section{Basics}

\subsection{Preliminaries and Notation}

For a set $S$ and object $o \notin S$ we write $S^o \eqdef S \uplus \set{o}$. For example, a domain $S$ can be lifted to its optional domain by writing $S^\bot$. 

\subsubsection{Functions}
\label{sec:functions}

For a (possibly partial) function $f : A \pfun B$ and $a \in A$ and $b \in B$, we write $\funup{f}{\ptup{a}{b}}$ for the updated function: \[ \funup{f}{\ptup{a}{b}} \eqdef \lambda x . \begin{cases}
	b & \text{if $x = a$} \\
	f(x) & \text{otherwise.}
\end{cases}\] We write $f(a) = \bot$ if the partial function $f$ is not defined at point $a$, i.e. if $a \notin \dom{f}$, and $\defined{f(a)}$ otherwise. When convenient, we also write $f_a$ as shorthand for $f(a)$. $a \mapsto b$ is shorthand for the partial function $f$ such that $f(a) = b$ and is undefined otherwise. For $A' \subseteq A$, $\restrict{f}{A'}$ is the restriction of $f$ to domain $A'$. 

For (possibly partial) functions $f,g : A \pfun B$, we we write $f \override g$ for the result of \emph{overriding} $f$ with $g$: \[ f \override g \eqdef \lambda x . \begin{cases}
	g(x) & \text{if $x \in \dom{g}$} \\
	f(x) & \text{otherwise.}
\end{cases} \] An obvious property is that, if $g(a) = b$, for some $b \in B$, then also $(f \override g)(a) = b$. Note also that if $f$ and $g$ have disjoint domains then $f \override g = f \uplus g$. 

\subsubsection{Lists}
\label{sec:lists}

The empty list is denoted by $\lnil$, the singleton list by $\lsingle{o}$, list construction by $o \lcons l$, and list concatenation by $l \lapp l'$, for object $o$ and lists $l,l'$. 

For a list $l$ of elements drawn from a set $A \times B$, we write $\funof{l}$ for the corresponding partial lookup function: \[ \funof{l} \eqdef \lambda x .\begin{cases}
	b &\text{ if $l = l' \lapp \lsingle{(x,b)}$}\\
	\funof{l'}(x) &\text{ if $l = l' \lapp \lsingle{(y,b)}$ with $x \neq y$}\\
	\bot & \text{otherwise.}
\end{cases}\] For $A' \subseteq A$, $\restrict{l}{A'}$ is the sublist restriction of $l$ to domain $A'$.

For convenience, we lift these function definitions pointwise to sets of lists. E.g., for a set $L$ of lists, $a \lcons L \eqdef \setof{a \lcons l}{l \in L}$. 

The set of \emph{interleavings} of lists $m,n$, written $m \merge n$, is defined by recursion on the structure of $m$ and $n$: \begin{align*}
	m \merge \lnil & = \lnil \merge m = \set{m} \\ 
	a \lcons m' \merge b \lcons n' & = a \lcons (m' \merge (b\lcons n')) \uplus b \lcons ((a \lcons m') \merge n').
\end{align*}

We now define a subset of the interleavings of lists of pairs from $A \times B$ that play an analogous role to to function overriding. The result of overriding a list $m$ with another $n$, written $m \override n$, is defined as follows: \[ l \in m \override n \iffdef \funof{l} = \funof{m} \override \funof{n}.\] As with function overriding, list overriding has the property that, if $\funof{n}(a) = b$, for some $b \in B$, and $l \in m \override n$, then $\funof{l}(a) = b$. Because all elements of $m \override n$ have the same lookup function, we may safely extend the list lookup notation as follows: \[ \funof{m \override n} \eqdef \lambda x . \funof{l}(x),\] for arbitrary $l \in m \override n$.

\subsection{Universes}

The various universal sets are declared and in some cases defined in Figure~\ref{fig:universes}. Note that, in the case of memory locations (i.e., addresses) and processor identifiers, 0 is excluded. 

\begin{figure}[ht]
	\centering
	\begin{tabular}{rcl|l}
		\multicolumn{3}{c}{Set} & Description \\ \hline
		\multicolumn{3}{l|}{$\setidentifiers$} & Identifiers \\
		$\setvalues$ & $=$ &  $\setintegers$ & Values \\
		$\setlocations$ & $\subseteq$  &  $\setpositives$ & Memory locations \\
		$\setprocessors$ &$\subseteq$ &  $\setpositives$ & Processor identifiers
	\end{tabular}
	\caption{\label{fig:universes}Universal Sets}
\end{figure}


\section{The Programming Language}
\label{sec:programming-language}

In this section I describe a simplified C-like structured programming language. The primitive commands closely resemble the basic memory events described by the memory model, while the composite commands are typical for high-level languages. In particular, note that this is not assembly language. This particular language of commands was chosen to be simple to reason about, but also at a suitable level of detail for describing concurrent data structures. Such algorithms are typically expressed using high level constructs like loops and if-then-else statements, along with basic atomic constructs like compare-and-swap, indication of where fencing is required, etc. 

\subsection{Expressions and Stacks}
\label{sec:expressions}

Expressions are terms that denote values, which in this development are just integers. Hence, they are also used later on to denote memory locations and processor identifiers. 

The language of expressions, written $\exprs$, is given by the following grammar: \[ \exprs~e \bnfdef v \bnfbar x \bnfbar (e + e') \bnfbar (e - e') \bnfbar \ldots, \] where $v \in \setvalues$ and $x \in \setidentifiers$. (The possibility of additional operations is left open; the complete set of expression is not important.)

The semantics of expressions is given w.r.t.~\emph{stacks}, which are total functions from $\setidentifiers$ to $\setvalues$. The collection of functions $\setidentifiers \tfun \setvalues$ is abbreviated $\setstacks$. The interpretation of an expression w.r.t.~a stack $s$ is given by the \emph{extension} of a stack, written $\ext{s}$, which is a total function from $\exprs$ to $\setvalues$ defined as follows: \begin{align*}
	\ext{s}(v) \eqdef & v \\
	\ext{s}(x) \eqdef & s(x) \\
	\ext{s}(e + e') \eqdef & \ext{s}(e) + \ext{s}(e') \\
	\ext{s}(e - e') \eqdef & \ext{s}(e) - \ext{s}(e')
\end{align*}  

Boolean expressions are terms that denote truth values. Their language is given by the following grammar: \[ \bexprs~b \bnfdef \bexpf \bnfbar \bexpt \bnfbar (!b) \bnfbar (e = e') \bnfbar \ldots, \] where $e,e' \in \exprs$. For convenience, we represent truth values by the set $\set{0,1}$ so the extension of a stack can also be used to interpret boolean expressions: \begin{align*}
	\ext{s}(\bexpf) \eqdef & 0 \\
	\ext{s}(\bexpt) \eqdef & 1 \\
	\ext{s}(!b) \eqdef & 1 - \ext{s}(b) \\
	\ext{s}(e = e') \eqdef & \begin{cases}
		1 &\text{ if } \ext{s}(e) = \ext{s}(e')\\
		0 &\text{ otherwise.}
	\end{cases}
\end{align*}

\subsection{Commands and States}

In this setting, programs are identified with \emph{commands}, which consist of compositions of \emph{primitive commands} for accessing and modifying state. In order to restrict the scope of the project, dynamic memory management commands (e.g., memory allocation and disposal) have been omitted.\footnote{I have no particular to reason to think that they will add significant complication though, and were also considered in earlier iterations of this project \cite{wmsldetails,lola11}.}

The syntax of primitive commands is given by the following grammar: \begin{align*} \pcomms~p \bnfdef & \cskip \bnfbar \cassume{b} \bnfbar \cassert{b} \bnfbar \cassign{x}{e} \bnfbar \cload{x}{e} \bnfbar \\ 
	& \cstore{e}{e'} \bnfbar \cfence \bnfbar \clock \bnfbar \cunlock 	
\end{align*}

The informal meaning of the primitive commands is as follows. \begin{itemize}
	\item $\cskip$ takes no evaluations steps;
	\item $\cassume{b}$ evaluates to $\cskip$ if $b$ holds and becomes stuck otherwise; 
	\item $\cassert{b}$ evaluates to $\cskip$ if $b$ holds and aborts otherwise;
	\item $\cassign{x}{e}$ assigns $e$ to identifier $x$; 
	\item $\cload{x}{e}$ assigns the value at memory address $e$ to identifier $x$; 
	\item $\cstore{e}{e'}$ stores the value $e'$ to memory address $e'$; 
	\item $\cfence$ commits any buffered writes to memory; 
	\item $\clock$ acquires the global lock and fences; and
	\item $\cunlock$ releases the global lock and fences.
\end{itemize}

The formal semantics of the successful execution of a primitive command $p$ is given as a transition relation between machine states $\sigma,\sigma'$ labeled by a processor identifier $i$ \[ p \vdash \sigma \stackrel{i}{\rightarrow} \sigma'. \] An informal reading of this quadruple indicates that when $p$ is executed on processor $i$ in state $\sigma$ it may evaluate to yield state $\sigma'$. (A formal interpretation will be given later in the context of a formal semantics of full commands.) A primitive command may alternatively abort upon execution: \[ p \vdash \sigma \stackrel{i}{\rightarrow} \top. \] For example, an assertion may fail or a process may attempt to access an unallocated (from the process's perspective) memory address. 

To defined these relations formally, we must first define the notion of a machine state. 

\begin{definition}
A partial machine state is a four-tuple $(s,h,B,k)$, where: \begin{itemize}
	\item $s : \setidentifiers \tfun \setvalues$ is a stack, as defined in Section~\ref{sec:expressions};
	\item $h : \setlocations \pfun \setvalues$ is a \emph{heap}, i.e., a partial function that represents the allocated locations of shared memory and their values; 
	\item $B : \setprocessors \tfun (\setlocations \times \setvalues)~\mathsf{list}$ is an array of write buffers; and
	\item $k : \powerset{\setprocessors}$ is a set of blocked processors. 
\end{itemize}
\end{definition} 
The collection of states is written $\setstates$. 

Note that the notion of machine state given here differs from that used to define the memory model. First, the set of names (i.e., ``registers,'' ``variables,'' ``identifiers,'' etc.) are global instead of local to each processor. This is for convenience only, and is not a technical restriction. The specification logic will be restricted to programs for which the names are partitioned among processes, except for those that are never modified to. Another reasonable choice would have been to use local names only, and to share read-only values among processes in the shared memory. This has the advantage of codifying the the above healthiness condition on programs directly into the model of the language and logic; it has the disadvantage of (perhaps) describing access to shared values slightly more awkward. 

Second, the global lock value from the memory model replaced in the machine model by a set of ``blocked'' processors, which are not allowed to access main memory. In the memory model an available lock indicates that no processors are blocked, while all lock held by processor $i$ indicates that all processors but $i$ are blocked. The machine model is more general, allowing an arbitrary subset of processors to be blocked. 

A partial state is called \emph{complete} if its set of blocked processors is a valid representation of a global lock, insofar as no processors are blocked when the lock is free, and all processors but $i$ are blocked when $i$ holds the lock: \[ \complete{s,h,B,k} \iffdef k = \nil \disj \exists i \in \setprocessors \st k = \setprocessors \setminus \set{i}.\]

The definition of the semantic relation for primitive commands is given in Figure~\ref{fig:primitive-semantics} below. 

\begin{figure}[p]
	\centering

	\begin{minipage}{\columnwidth}

		\infrule[p-assume]{\text{ if  $\ext{s}(b) = 1$}}{\cassume{b} \vdash (s,h,B,k) \stackrel{i}{\rightarrow} (s,h,B,k)}

		\vspace{1em}

		\infrule[p-assert]{\text{ if  $\ext{s}(b) = 1$}}{\cassert{b} \vdash (s,h,B,k) \stackrel{i}{\rightarrow} (s,h,B,k)}

		\vspace{1em}

		\infrule[p-assert-a]{\text{ if  $\ext{s}(b) = 0$}}{\cassert{b} \vdash(s,h,B,k) \stackrel{i}{\rightarrow} \top}

		\vspace{1em}

		\infrule[p-assign]{}{\cassign{x}{e} \vdash(s,h,B,k) \stackrel{i}{\rightarrow} (\funup{s}{\ptup{x}{\ext{s}(e)}},h,B,k)}

		\vspace{1em}

		\infrule[p-load]{\text{if $(h \override \funof{B_i})(\ext{s}(e)) = v$ and $i \notin k$}}{\cload{x}{e} \vdash(s,h,B,k) \stackrel{i}{\rightarrow} (\funup{s}{\ptup{x}{v}},h,B,k)}

		\vspace{1em}


		\infrule[p-load-a]{\text{if $(h \override \funof{B_i})(\ext{s}(e)) = \bot$ and $i \notin k$}}{\cload{x}{e} \vdash(s,h,B,k) \stackrel{i}{\rightarrow} \top}
		
		\vspace{1em}

		\infrule[p-store]{\text{if $\ext{s}(e) \in \dom{h \override \funof{B_i}}$}}{\cstore{e}{e'} \vdash(s,h,B,k) \stackrel{i}{\rightarrow} (s,h,\funup{B}{\ptup{i}{B_i \lapp \lsingle{\ext{s}(e),\ext{s}(e')}}},k)}
		
		\vspace{1em}

		\infrule[p-store-a]{\text{if $\ext{s}(e) \notin \dom{h \override \funof{B_i}}$}}{\cstore{e}{e'} \vdash(s,h,B,k) \stackrel{i}{\rightarrow} \top}
		
		\vspace{1em}

		\infrule[p-fence]{\text{if $B_i = \lnil$}}{\cfence \vdash(s,h,B,k) \stackrel{i}{\rightarrow} (s,h,B,k)}
		
		\vspace{1em}

		\infrule[p-lock]{\text{if $B_i = \lnil$}}{\clock \vdash(s,h,B,\nil) \stackrel{i}{\rightarrow} (s,h,B,\set{i})}
		
		\vspace{1em}

		\infrule[p-unlock]{\text{if $B_i = \lnil$}}{\cunlock \vdash(s,h,B,\set{i}) \stackrel{i}{\rightarrow} (s,h,B,\nil)}
	\end{minipage}
	\caption{\label{fig:primitive-semantics} Semantics of primitive commands}
\end{figure}

Structured commands consist of either a primitive command; a sequential or concurrent composition of commands; a nondeterministic choice between commands; or an iteration of a command. We assume that, as commands, primitive commands are annotated with the name of the processor on which they are to be executed. Presumably, the components of a sequential command will all be scheduled to execute on the same processor, but this is not required and the semantics handles all cases uniformly. This generality is not likely to be practically useful for sequential composition, but is crucial to the semantics of concurrent composition, as will be discussed shortly. 

The language of commands is defined by the following grammar: \[ \comms~c \bnfdef p_e \bnfbar (\cseq{c}{c'}) \bnfbar (\cchoice{c}{c'}) \bnfbar \cloop{c} \bnfbar (\cpar{c}{c'}),\] where $p$ is a primitive command and $e$ is an expression which (hopefully) denotes a processor identifier.  

The formal semantics of the successful execution of a command is given as a binary transition relation between command-state pairs: \[ \vdash c,\sigma \rightarrow c',\sigma'.\] But a command's execution may abort unsuccessfully as well, as with primitive commands. Unsuccessful executions are modeled as a transition relation between command-state pairs and an erroneous pseudo-state, as for primitive commands: \[ \vdash c,\sigma \rightarrow \top. \] We refer collectively to command-state pairs and the erroneous state $\top$ as \emph{configurations}, and use $\mathcal{C}$ to indicate a configuration. 

The semantics of commands also encompasses ``silent'' transitions, which represent the flushing of buffered writes to the shared memory as allowed by the memory model. This flushing is described by a relation between states $\stackrel{\tau}{\rightarrow}$ defined as follows: \begin{align*} (s,h,B,k) \stackrel{\tau}{\rightarrow} (s,\funup{h}{\ptup{\ell}{v}},\funup{B}{\ptup{i}{b}},k) \iffdef & B_i = \lsingle{(\ell,v)} \lapp b \conj i \notin k 
\end{align*} We write $\taurefines$ for the reflexive-transitive closure of $\stackrel{\tau}{\rightarrow}$. 

The complete relation that defines the semantics of commands is given in Figure~\ref{fig:command-semantics} below.

\begin{figure}[ht]
	\centering

	\infrule[c-prim]{\text{ if $p \vdash \sigma \stackrel{\ext{s}(e)}{\rightarrow} \sigma'$ and $\sigma' \in \setstates$}}{\vdash p_e,\sigma \rightarrow \cskip,\sigma'}

	\vspace{1em}

	\begin{tabular}{ll}
	\begin{minipage}{.43\columnwidth}

		\infrule[c-prim-a]{\text{ if $p \vdash \sigma \stackrel{\ext{s}(e)}{\rightarrow} \top$}}{\vdash p_e,\sigma \rightarrow \top}

		\vspace{1em}

		\infrule[c-tau]{\text{if $\sigma \stackrel{\tau}{\rightarrow} \sigma'$}}{\vdash c,\sigma \rightarrow c,\sigma'}

		\vspace{1em}

		\infrule[c-seq]{\vdash c,\sigma \rightarrow c_0,\sigma'}{\vdash (\cseq{c}{c'}),\sigma \rightarrow (\cseq{c_0}{c'}),\sigma'}

		\vspace{1em}

		\infrule[c-seq-a]{\vdash c,\sigma \rightarrow \top}{\vdash (\cseq{c}{c'}),\sigma \rightarrow \top}

		\vspace{1em}

		\infrule[c-seq-s]{}{\vdash (\cseq{\cskip}{c'}),\sigma \rightarrow c',\sigma}

		\vspace{1em}

		\infrule[c-ch-1]{}{\vdash (\cchoice{c}{c'}),\sigma \rightarrow c,\sigma}

		\vspace{1em}

		\infrule[c-ch-2]{}{\vdash (\cchoice{c}{c'}),\sigma \rightarrow c',\sigma}

\end{minipage} & 
\begin{minipage}{.52\columnwidth}

		\infrule[c-par-1]{\vdash c,\sigma \rightarrow c_0,\sigma'}{\vdash (\cpar{c}{c'}),\sigma \rightarrow (\cpar{c_0}{c'}),\sigma'}

		\vspace{1em}

		\infrule[c-par-1a]{\vdash c,\sigma \rightarrow \top}{\vdash (\cpar{c}{c'}),\sigma \rightarrow \top}

		\vspace{1em}

		\infrule[c-par-1s]{}{\vdash (\cpar{\cskip}{c'}),\sigma \rightarrow c',\sigma}

		\vspace{1em}

		\infrule[c-par-2]{\vdash c',\sigma \rightarrow c_0,\sigma'}{\vdash (\cpar{c}{c'}),\sigma \rightarrow (\cpar{c}{c_0}),\sigma'}

		\vspace{1em}

		\infrule[c-par-2a]{\vdash c',\sigma \rightarrow \top}{\vdash (\cpar{c}{c'}),\sigma \rightarrow \top}

		\vspace{1em}

		\infrule[c-par-2s]{}{\vdash (\cpar{c}{\cskip}),\sigma \rightarrow c,\sigma}

		\vspace{1em}

		\infrule[c-loop]{}{\vdash \cloop{c},\sigma \rightarrow (\cchoice{\cskip}{(\cseq{c}{\cloop{c}})}),\sigma}

\end{minipage}
\end{tabular}
	\caption{\label{fig:command-semantics} Semantics of commands}
\end{figure} 

The reflexive-transitive closure of command evaluation semantics, written $\vdash c,\sigma \rightarrow^\ast \mathcal{C}$, is defined as usual. 

\paragraph{Stability} Consider a state $\sigma_0 = (s,h,B,\nil)$ in which $h = \ell \mapsto 0$, $B(j) = \lsingle{(\ell,1)}$ and $B(x) = \lnil$ for all $x \neq j$. From this state, a load on processor $i$ may evaluate as follows: \[ \vdash \cload{y}{\ell}_i,(s,h,B,\nil) \rightarrow \cskip,(\funup{s}{\ptup{y}{0}},h,B,\nil).\] Because $j$ is not blocked, it is also possible for a flushing operation to take place: \[ \vdash \cload{y}{\ell}_i,(s,h,B,\nil) \rightarrow \cload{y}{\ell}_i,(s,\funup{h}{\ptup{\ell}{1}},\funup{B}{\ptup{j}{\lnil}},\nil),\] and afterward for the load to evaluate as follows: \[ \vdash \cload{y}{\ell}_i,(s,\funup{h}{\ptup{\ell}{1}},\funup{B}{\ptup{j}{\lnil}},\nil) \rightarrow \cskip(\funup{s}{\ptup{y}{1}},\funup{h}{\ptup{\ell}{1}},\funup{B}{\ptup{j}{\lnil}},\nil).\] Note that in the first evaluation the load resolves $\ell$ to $0$, and in the second evaluation it resolves $\ell$ to $1$, with the distinguishing characteristic of the latter being the preceding nondeterministic flushing operation. 

By contrast, from the state $\sigma_1 = (s,h,B,\set{j})$, where $h$ and $B$ are defined as in $\sigma_0$, the \emph{only} reduction of $\cload{y}{\ell}_i$ is: \[ \vdash \cload{y}{\ell}_i,(s,h,B,\nil) \rightarrow \cskip,(\funup{s}{\ptup{y}{0}},h,B,\nil).\] This is because processor $j$ is blocked, and so its buffered write may not commit to memory. As a consequence, it is not possible for the load on processor $i$ to observe the write buffered on processor $j$. 

We say that location $\ell$ is \emph{unstable} in state $\sigma_0$ for processor $i$ because the result of loading $\ell$ is determined by the relative ordering of the flushing operations. On the other hand, $\ell$ is \emph{stable} in $\sigma_1$ for $i$ because the load of $\ell$ is oblivious to the flushing operations. 

A state is called \emph{coherent} if each location has writes buffered by at most one processor: \[ \forall i,j \in \setprocessors \setminus k \st i \neq j \onlyif \dom{B(i)} \cap \dom{B(j)} = \nil.\] The memory locations in a coherent state may be partitioned among the processors, such that the locations of a partition are all stable for their respective processor.

\paragraph{Interleaving versus Parallelism} A pleasant property of this semantics is the uniform description of both interleaving and parallel concurrency. Let $c_i$ be a sequential command $c$ in which each primitive has processor annotation $i$. Then, e.g., $(\cpar{c_1}{c'_1})$ describes the interleaving concurrent execution of commands $c$ and $c'$ on processor 1, while $(\cpar{c_1}{c'_2})$ describes the parallel concurrent execution of $c$ and $c'$ on processors 1 and 2, respectively. But one does not typically have control over the particular processor on which a command executes (e.g., $c_1$ instead of $c_2$). Thus, $(\cpar{c_x}{c'_x})$ describes the interleaving concurrent execution of commands $c$ and $c'$ on some individual processor, denoted by the free variable $x$. For $x \neq y$, $(\cpar{c_x}{c'_y})$ describes the parallel concurrent execution of $c$ and $c'$ on distinct processors given by $x$ and $y$ respectively. Furthermore, without any assumptions about the relationship between $x$ and $y$, $(\cpar{c_x}{c'_y})$ describes both interleaving and parallel executions of $c$ and $c'$. This presumably is the most common situation with concurrent composition: it is up to the operating system to assign processors to threads, and correctness of a program ought to encompass any such assignment. 
\paragraph{Command Abbreviations} A few standard command abbreviations are shown in Figure~\ref{fig:command-abbreviations}. Some would benefit greatly from local variable declarations, which I have not yet added to the language. 

\begin{figure}[ht]
	\centering
	\resizebox{\columnwidth}{!}{
	\begin{minipage}{\columnwidth}
	\begin{align*}
		\cifthenelse{b}{c}{c'} \eqdef & \cchoice{(\cseq{\cassume{b}}{c})}{(\cseq{\cassume{!b}}{c'})} \\
		\cifthen{b}{c} \eqdef & \cchoice{(\cseq{\cassume{b}}{c})}{(\cseq{\cassume{!b}}{\cskip})} \\
		\cwhile{b}{c} \eqdef & \cseq{\cloop{(\cseq{\cassume{b}}{c})}}{\cassume{!b}} \\
		\cwith{e}{c} \eqdef & \cseq{\clock_{e}}{\cseq{c}{\cunlock_{e}}} \\ 
		\mathsf{inc}(e,e') \eqdef & \cwith{e}{(\cseq{\cload{x}{e'}}{\cstore{e'}{x+1}})} \\
		\ccas{e,f,g,g'} \eqdef & \cwith{e}{(\cseq{\cload{x}{f}}{\cifthenelse{x = g}{\cstore{f}{g'}}{\cassign{r}{0}}})}
	\end{align*}
\end{minipage}}
	\caption{\label{fig:command-abbreviations} Command abbreviations}
\end{figure}

\paragraph{Static Semantics} The static semantics of expressions, primitive commands and commands, embodied here by functions $\fv{-}$ and $\mod{-}$ associating these objects to their sets of free and modified variables, respectively, are completely standard. (Especially so because there is are no name-hiding operations in the language, like the aforementioned missing local variable declaration command.) E.g., $\fv{\cload{x}{y+1}_z} = \set{x,y,z}$ and $\mod{\cload{x}{y+1}_z} = \set{x}$.  

\section{Assertions}
\label{sec:assertions}

Assertions denote sets of machine states, and are used to write the pre- and post-conditions of commands in the specification logic. The language is defined by the following grammar: \begin{align*}
	\asserts~P \bnfdef & b \bnfbar (P \disj P') \bnfbar (P \conj P') \bnfbar (\exists x \st P) \bnfbar (\forall x \st P) \bnfbar \\
	& \femp \bnfbar \fbar{e} \bnfbar \flock{e} \bnfbar e \fwrite{e'} e'' \bnfbar  \\ 
	& (P \fhash P') \bnfbar (P \fsep P') \bnfbar (P \fseq P') \bnfbar (P \fsseq P')
\end{align*} 

The informal meaning of these assertions are as follows. The lifting of a boolean expression to an atomic formula, disjunction, conjunction and quantification have the usual meaning. $\femp$ describes states with an empty heap and all write buffers empty. $\fbar$ describes states in which just the $e$th buffer is empty. $\flock{e}$ asserts that processor $e$ holds the lock. $e \fwrite{e'} e''$ describes a single write to location $e$ with value $e''$, either buffer on processor $e$ or flushed to memory. $(P \fseq P')$ describes per-write buffer concatenation of writes. $(P \fsseq P')$ is like $(P \fseq P')$, but in which disjointness of allocated locations is required. $(P \fsep P')$ also requires disjointness, but interleaves the described writes on each write buffer instead of concatenating them. Finally, $(P \fhash P')$ is weaker than the three other separating conjunctions, and provides the ``most general'' frame rule. 

The set of free variables of an assertion, written $\fv{P}$, is defined as usual. 

\paragraph{Assertion Abbreviations} The following abbreviation, analogous to the points-to formula of separation logic, describes the result of flushing a single write to memory: \begin{align*}
	e \fpointsto e' \eqdef & \exists x \st e \fwrite{x} e' \conj \fbar{x}\\
\end{align*} (We will note later that $(e \fwrite{x} e' \conj \fbar{x})\sequiv e \fwrite{y} e' \conj \fbar{y}$, which justifies this notation.) 

\subsection{Predicates}
\label{sec:predicates}

Assertions denote particular sets of intensional machine states, which we refer to as predicates. In particular, a predicate is a set of states $S$ that satisfies the following flushing-closure property: if $\sigma \in S$ and $\sigma \stackrel{\tau}{\rightarrow} \sigma'$ then $\sigma' \in S$. The property is needed for soundness w.r.t.\ the memory model, which allows non-blocked processors to commit buffered writes to memory nondeterministically.  

\subsection{Separation and Locality}

We now digress to define a series of functions on states, which will later be used to define the predicates for each of the four separating conjunctions. With these definitions in hand, the predicate definitions will be straightforward.

An explicit goal is that each conjunction $\bullet$ be \emph{local} w.r.t.\ the programming language in the following way. Suppose we wish to show that, when a command $c$ is evaluated in a complete state $\Sigma_0$, that it does not abort, and that the resulting state $\Sigma_1$ has some some property $P$. That is, we wish to show that $c,\Sigma_0 \rightarrow^\ast \cskip,\Sigma'$ and $P(\Sigma_1)$ holds. 

In some scenarios, we may attack the problem as follows. Suppose the property $P$ only depends on some particular substate of a complete state $\Sigma$, with the remaining portion---the \emph{frame}---irrelevant to the property at hand. For example, the property may relate the values of only a few particular memory locations and is independent of the others. Let us write $\Sigma = \sigma_F \bullet \sigma$ to denote the decomposition of a complete state $\Sigma$ into substates $\sigma_F$ and $\sigma$. For example, perhaps $\sigma$ describes the relevant memory locations, and $\sigma_F$ the irrelevant locations. Then it is the case that $P(\sigma)$ implies $P(\Sigma)$. 

Consider again our task of showing that, for some $\Sigma_1$, $c,\Sigma_0 \rightarrow^\ast \cskip,\Sigma_1$ and $P(\Sigma_1)$. If we know that that the property depends just on a particular substate, we may try to reason about the behavior of the command with respect to just the substate that is relevant to the property. We proceed then by decomposing the initial state $\Sigma_0$ into a relevant substate $\sigma_0$ and a frame state $\sigma_F$, with $\Sigma_0 = \sigma_F \bullet \sigma_0$, and consider the behavior of the command $c$ in the substate $\sigma_0$. 

Perhaps we are able to show that $c$ does not abort in this substate, and evaluates to some state $\sigma_1$: $c,\sigma_0 \rightarrow^\ast \cskip,\sigma_1$. Can we conclude that the $c$ also does not abort in the complete state $\Sigma$? If the command obeys what is known as the \emph{safety monotonicity} property, then we may indeed conclude this. That is, safety monotonicity asserts that if a command executes successfully in some state, then it also execute successfully in a superstate.  

Suppose also that we can show that the desired property holds of the state that results from the evaluation of the command in the substate $\sigma_0$; i.e., that $P(\sigma_1)$ holds. Because we know that the frame substate is irrelevant to the property $P$, it is also the case that $P(\sigma_F \bullet \sigma_1)$ holds. But what is the relationship between $\Sigma_1$ and the composite state $\sigma_F \bullet \sigma_1$? If the command obeys what is known as the \emph{frame property}, then in fact $\Sigma_1 = \sigma_F \bullet \sigma_1$. That is, the frame property asserts that whenever $\Sigma_0 = \sigma_F \bullet \sigma_0$ and $c,\Sigma_0 \rightarrow^\ast \Sigma_1$, and if $c$ does not abort on the substate $\sigma_1$, then there exists some $\sigma_1$ such that $\Sigma_1 = \sigma_F \bullet \sigma_1$. Hence, $P(\Sigma_1)$ holds. 

This is the essence of \emph{local reasoning}: to reason about the behavior of a command in a complete state, we decompose the complete state into a local part and a frame part, reason only about the behavior of the command with respect to the local part, and finally draw conclusions about the behavior of the command in the complete state based on this local reasoning. A command that obeys the safety monotonicity and frame properties---with respect to particular notion of decomposition---is said to be \emph{local}.

\subsubsection{Weak Interleaving Separation}

We begin by defining a semantic function called \emph{weak interleaving separation}, so-called because it shall not require disjointness of the memory locations described by its arguments (otherwise it would be \emph{strong}), and because the buffered writes described by its arguments are, roughly speaking, interleaved (as opposed to, say, concatenated). 

We write $\sigma \fhash \sigma'$ for the weak interleaving separation of $\sigma$ and $\sigma'$. The domain of the function is a set of states because interleaving write buffers results in a set of possible interleaved write buffers. This separation function is intended to be as weak as possible while still maintaining locality w.r.t.\ the programming language. This weakness reduces the expressive power of the function, but later on we will define stronger, more expressive functions by strengthening this function in a variety of ways. 

We begin by lifting the the \emph{overriding} operation---as in the overriding of functions (Section~\ref{sec:functions}) and lists (Section~\ref{sec:lists})---to states. First, we lift list overriding to functions into lists: \[ B \in B_1 \override B_2 \iffdef \forall i \in \dom{B} \st B(i) \in B_1(i) \override B_2(i). \] Overriding $\sigma_1 = (s,h_1,B_1,k_1)$ by $\sigma = (s_2,h_2,B_2,k_2)$, is then given by $\sigma_1 \override \sigma_2$: \[ \sigma_1 \fhash \sigma_2 \eqdef \setof{(s,h_1 \override h_2, B,k_1 \uplus k_2)}{B \in B_1 \override B_2},\] The \emph{weak interleaving separation} of states is defined as the overriding of compatible states: \[ \sigma_1 \fhash \sigma_2 \eqdef \begin{cases}
	\sigma_1 \override \sigma_2 & \text{if $\sigma_1 \compat \sigma_2$} \\ 
	\nil & \text{otherwise,} \end{cases} \] where the compatibility relation $\sigma_1 \compat \sigma_2$ is defined as the conjunction of the following conditions: \begin{enumerate}
	\item $k_1 \cap k_2 = \nil$,
	\item $\forall i \in \setprocessors \setminus (k_1 \cup k_2) \st \dom{B_1(i)} \cap \dom{h_2} = \nil$, 
	\item $\forall i,j \in \setprocessors \setminus (k_1 \cup k_2) \st i \neq j \onlyif \dom{B_1(i)} \cap \dom{B_2(j)}$. 
\end{enumerate}

For convenience, we overload the symbol $\fhash$ to indicate the pointwise lifting of this function to sets of states: \[ S_1 \fhash S_2 \eqdef \cup \setof{s_1 \fhash s_2}{s_1 \in S_1 \conj s_2 \in S_2 \conj s_1 \compat s_2}.\] We use these functions interchangeably when the intended meaning is clear from context, e.g.: \[ s_1 \fhash (s_2 \fhash s_3) = \cup \setof{s_1 \fhash s_{23}}{s_{23} \in s_2 \fhash s_3}.\]


\subsubsection{Weak Interleaving Separation Old}

We begin by defining a partial semantic function for the most general separating conjunction, written $\sigma \fhash \sigma'$, from which the others are derived. This operation will be particularly useful for giving meaning to program specifications, while the other, stronger conjunctions will be more useful for expressing the axioms for commands in the specification logic. 

We wish for commands to be local w.r.t.~this conjunction so that it may have a corresponding frame rule. By way of example, consider a load $\cload{x}{\ell}$ on processor $i$ in a state $\sigma$. Let us consider the manners in which the state $\sigma$ can be extended while preserving the essential behavior of the command---viz.~the resultant value of the load. \begin{enumerate}
	\item We \emph{may} augment $\sigma$ with additional buffered writes to address $\ell$ on buffer $i$ if those writes occur before the most recent writes to $\ell$ on $i$. The load command only returns the most recent buffered write, so additional earlier buffered writes will not affect the result. 
	
	\item We \emph{may not} augment $\sigma$ with additional buffered writes to address $\ell$ on $i$ that are more recent than the those already present, for these additional writes certainly will affect the outcome of the load. 
	
	\item We \emph{may} augment $\sigma$ with additional committed writes to address $\ell$ if those writes again precede previously committed writes to $\ell$ in $\sigma$. (Because committed writes implicitly precede all buffered writes, this is consistent with the first scenario in which the $i$th buffer is safely augmented with earlier writes.)
	
	\item We \emph{may} augment $\sigma$ with additional writes to locations distinct from $\ell$, either committed to memory or buffered on $i$, regardless of their ordering with respect to writes to $\ell$ already present. The resultant value of the load command is not affected by writes to the locations not being loaded.  
	
	\item We \emph{may} augment $\sigma$ with additional writes to locations distinct from $\ell$ on other buffers as well. Although those writes may commit before or after the $\ell$-writes being loaded by $i$, they do not affect the result. 
	
	\item Finally, consider writes to address $\ell$ on buffer $j$ with $j \neq i$. In general, this may adversely affect the load on $i$ because we are unable to predict the order in which the writes buffered by $j$ will commit to memory. It is possible that they will commit after buffered writes on $i$ have committed but before the load has completed, thus affecting the result of the load. So it would seem that such writes must be disallowed. 
	
	There is however a case in which it is safe to augment $\sigma$ in this way: namely, when buffer $j$ is blocked. For in this case writes buffered by $j$ will not be committed to memory, and so there is no risk that these writes will be made visible to the load on $i$. Hence, we \emph{may} augment other buffers with writes to $\ell$ when $j$ is blocked. 

	Furthermore, it is safe to augment the state with $\ell$-writes to other buffers when the lock is held by \emph{any} processor, and not just by $i$. For if $j$, with $j \neq i$, holds the lock, then a load on processor $i$ cannot proceed! And so, from a partial correctness standpoint, the writes on other processors are quite irrelevant.
\end{enumerate}

We now define a function motivated by these scenarios, which maps pairs of states into a set of states. This is accomplished by specifying with which states $\sigma_0$ a given state $\sigma$ may be augmented, and by defining the possible results of this augmentation as the set of states $\sigma_0 \fhash \sigma$. Note that because some of the preceding scenarios are asymmetric---e.g., we may augment $\sigma$ with earlier writes to a particular location, but not later writes---the resulting operation will not be commutative. This is, of course, in contrast to the semantic function for the separating conjunction of traditional separation logic. 

We begin by proposing a definition for $\sigma \fhash \sigma'$, and afterward determine compatibility between states. $\sigma \fhash \sigma'$ is defined approximately as the product of operations; one of that combines the heaps of $\sigma$ and $\sigma'$, and one that combines the individual buffers of $\sigma$ and $\sigma'$. These operations shall only constrain the order among the writes in $\sigma$ and $\sigma'$ when necessary.  Furthermore, both operations are essentially the same: the \emph{overriding} of heaps and buffers, as defined in Sections~\ref{sec:functions} and~\ref{sec:lists}, respectively. From the scenarios above, we see that the only additional ordering constraints required are between writes to the same address, which is exactly what is provided by the overriding operations. 

The proposed definition is as follows: 
\[ (s,h_1,B_1,k_1) \fhash (s,h_2,B_2,k_2) \eqdef \bigcup_{B \in B_1 \override B_2} \set{(s,h_1 \override h_2, B,k_1 \uplus k_2)},\] where $B_1 \override B_2$ indicates list overriding lifted pointwise to functions: \[ B \in B_1 \override B_2 \iffdef \forall i \in \dom{B} \st B(i) \in B_1(i) \override B_2(i). \]

Let us check to see whether this definition is consistent with the scenarios above: \begin{enumerate}
	\item Buffered $\ell$-writes on buffer $i$ necessarily precede any $\ell$-writes already present on buffer $i$ by definition of buffer overriding, $B_i \override B'_i$, consistent with the corresponding scenario above. 
	
	\item Buffered writes to other locations on buffer $i$ are unordered w.r.t.~existing writes by definition of buffer overriding, $B_i \override B'_i$, consistent with the corresponding scenario above.
	
	\item Committed $\ell$-writes necessarily precede any $\ell$-writes already present in the heap by definition of heap overriding, $h \override h'$, consistent with the corresponding scenario above. 
	
	\item Committed writes to other locations are unordered w.r.t.~writes already present in the heap by definition of heap overriding $h \override h'$, consistent with the corresponding scenario above. 
	
	\item There are no ordering constraints among buffered writes on different write buffers by definition of overriding write buffer arrays, $B \fhash B'$. So in case the additional writes are to locations different from the already present writes, the definition is consistent with the corresponding scenario above. 
	
	\item Unfortunately, in the case the additional buffered writes are to locations already present on in other buffers, the definition is \emph{not} consistent with the corresponding scenario above, because $\ell$ writes may added to buffer $j$ even when $\ell$ writes already exist in buffer $i$. 
\end{enumerate}

We begin by defining \emph{compatibility} between states, written $\sigma \compat \sigma'$. For $\sigma = (s,h,B,k)$ and $\sigma' (s',h',B',k')$, the following four conditions must be satisfied for compatibility: \begin{enumerate}
	\item $s = s'$,
	\item $k \cap k' = \nil$,
	\item $\forall j \in \setprocessors \setminus (k \cup k') \st \dom{B_j} \cap \dom{h'} = \nil$, 
	\item $\forall i,j \in \setprocessors \setminus (k \cup k') \st i \neq j \onlyif \dom{B_j} \cap \dom{B_i}$. 
\end{enumerate}

To recover consistency, we must rule out the case in which the additional buffered writes to location $\ell$ on processor $i$ are ruled out if when writes to $\ell$ are already present in the heap, or already present in a buffer other than $i$. But, if $i$ is not live, then we need not rule out such additional writes. And we need only concern ourselves with conflicts between additional writes on $i$ and already present writes on another processor $j$ if both $i$ and $j$ are live; i.e., if the lock is available. These two conditions---ruling out conflicts among additional buffered writes with \emph{1)} the existing heap, and \emph{2)} the existing buffered writes---can be described as follows: \begin{align*}
	\forall j \in  \live{l} \st \dom{B_j} \cap \dom{h'} & = \nil \\
	\forall j \in  \live{l} \st \forall i \in \live{l} \st i \neq j \onlyif \dom{B_j} \cap \dom{B'_i} & = \nil. \\
\end{align*} Or, more succinctly, as a single condition: \[ \forall j \in \live{l} \st \dom{B_j} \cap \left( \dom{h'} \cup \bigcup_{i \in \live{l} \st i \neq j} \dom{B'_i}\right) = \nil. \] 
This, along with the requirement that the stacks and locks of the two states agree, is the definedness condition for $\sigma \fhash \sigma'$.

We refer to this as \emph{weak interleaving separation}, because only minimal ordering is created among the writes in the conjoined states (and unordered writes represent their various interleavings), and because memory addresses are not partitioned, but instead only weakly separated. 

\begin{lemma}
	\label{lem:separation-tau}
	If $\sigma \in \sigma_1 \fhash \sigma_2$ and $\sigma \stackrel{\tau}{\rightarrow} \sigma'$, then either there exists $\sigma'_1$ such that $\sigma' \in \sigma'_1 \fhash \sigma_2$, or there exists $\sigma'_2$ such that $\sigma' \in \sigma_1 \fhash \sigma'_2$.
\end{lemma}

\begin{proof}
	Without loss of generality, assume $\sigma = (s,h,\funup{B}{\ptup{i}{(\ell,v)\lcons b}},k)$ and thus $\sigma' = (s,\funup{h}{\ptup{\ell}{v}},\funup{B}{\ptup{i}{b}},k)$, with $i \notin k$. Furthermore, we have $h = h_1 \override h_2$, $B(i) = (\ell,v) \lcons b \in B_1(i) \override B_2(i)$ and $k = k_1 \uplus k_2$, assuming $\sigma_1 = (s,h_1,B_1,k_1)$ and $\sigma_2 = (s,h_2,B_2,k_2)$. The least-recent write of $B(i)$, $(\ell,v)$, is either the least-recent write of $B_1(i)$ or $B_2(i)$. 

	In the first case, $B_1(i) = (\ell,v) \lcons b_1'$, and $b \in b'_1 \override B_2(i)$. Let $\sigma'_1 = (s,\funup{h_1}{\ptup{\ell}{v}},\funup{B_1}{\ptup{i}{b'_1}},k_1)$. Because $i \notin k = k_1 \uplus k_2$, it is also the case that $i \notin k_1$, and so $\sigma_1 \stackrel{\tau}{\rightarrow} \sigma'_1$. By definedness of $\sigma_1 \fhash \sigma_2$, we know that if $\ell \in \dom{h_2} \cap \dom{B_1(i)}$ then $i \in k$. Hence $\ell \notin \dom{h_2}$, which means that $\funup{(h_1 \override h_2)}{\ptup{\ell}{v}} = \funup{h_1}{\ptup{\ell}{v}} \override h_2$. It follows that $\sigma' = (s,\funup{h_1}{\ptup{\ell}{v}} \override h_2, \funup{B}{\ptup{i}{b}},k) \in \sigma'_1 \fhash \sigma_2$. 

	In the second case, $B_2(i) = (\ell,v) \lcons b'_2$, and $b \in B_1(i) \override b'_2$. Let $\sigma'_2 = (s,\funup{h_2}{\ptup{\ell}{v}},\funup{B_2}{\ptup{i}{b'_2}},k_2)$. Again, $i \notin k_2$, and so $\sigma_2 \stackrel{\tau}{\rightarrow} \sigma'_2$. Because $\funup{(h_1 \override h_2)}{\ptup{\ell}{v}} = h_1 \override (\funup{h_2}{\ptup{\ell}{v}})$, it follows that $\sigma' = (s,h_1 \override (\funup{h_2}{\ptup{\ell}{v}}),\funup{B}{\ptup{i}{b}},k) \in \sigma_1 \fhash \sigma_2$. 
\end{proof}

\begin{lemma}
	\label{lem:separation-refinement}
	If $\sigma \in \sigma_1 \fhash \sigma_2$ and $\sigma' \taurefines \sigma$ then there exists $\sigma'_1,\sigma'_2$ such that $\sigma'_1 \taurefines \sigma_1$, $\sigma'_2 \taurefines \sigma_2$ and $\sigma' \in \sigma'_1 \fhash \sigma'_2$. 
\end{lemma}

\begin{proof}
	By induction on the number of $\tau$ steps from $\sigma$ to $\sigma'$. The base case is trivial. Otherwise, assume that $\sigma'' \taurefines \sigma$ and $\sigma'' \stackrel{\tau}{\rightarrow} \sigma'$, and by the induction hypothesis that there exists $\sigma''_1,\sigma''_2$ such that $\sigma''_1 \taurefines \sigma_1$, $\sigma''_2 \taurefines \sigma_2$ and $\sigma'' \in \sigma''_1 \fhash \sigma''_2$. In which case the result follows from Lemma~\ref{lem:separation-tau} and transitivity of $\taurefines$.
\end{proof}


It is possible to imagine strengthening this conjunction in two ways: \begin{enumerate}
	\item With stronger disjointness requirements, barring \emph{any} overlap in the addresses to which there are writes between the conjuncts, regardless of the lock status and the presence of ``covering'' writes in the present state.  
	
	\item With stronger ordering properties, yielding a state in which all writes added precede all existing writes, instead of one in which only ordering between writes to the same address is created. Weak interleaving separation is unable to express ordering constraints among writes to the same location on a single buffer. 
\end{enumerate}

The next three subsections describe separating conjunctions that embrace one or both of these additional properties. 

\subsubsection{Strong Interleaving Separation}

The strong interleaving separation of states, written $\sigma \fsep \sigma'$, is like the weak interleaving separation, but in which the domains of memory addresses are required to be completely disjoint. Later, this separation will be useful for describing the relationship between local and shared state, which we will require to be disjoint regardless of the status of the lock, and for describing concurrent composition of commands.

 The definedness condition for $\sigma \fsep \sigma$ (besides agreement on stacks and locks) is as follows: \[ \left(\dom{h} \cup \bigcup_{i \in \setprocessors} \dom{B_i} \right) \cap \left( \dom{h'} \cup \bigcup_{i \in \setprocessors} \dom{B'_i}\right) = \nil, \] and the value of $\sigma \fsep \sigma'$ is as follows: \[ (s,h,B,l) \fsep (s,h',B',l) \eqdef (s,h\fhash h',B\fhash B',l).\] 

Because the domains the heaps and buffers are disjoint, this is equivalent to the following definition that makes use of the simpler additive union in place of the TSO union: \[ (s,h,B,l) \fsep (s,h',B',l) \eqdef (s,h\uplus h',B\uplus B',l).\] 

Observe that if $\sigma \fsep \sigma'$ is defined then $\sigma \fhash \sigma'$ is defined and identical the former state. 

\subsubsection{Weak Sequential Separation}

The weak sequential separation of states, written $\sigma \fseq \sigma'$, is like the weak interleaving separation, but in which all the additional writes in $\sigma$ precede writes already present in $\sigma'$. With weak interleaving separation, the additional writes were only required to precede those already present with the same address. 

The value, when defined, of $\sigma \fseq \sigma'$ is like $\sigma \fhash \sigma'$ but uses pomset concatenation instead of the TSO union: \[ (s,h,B,l) \fseq (s,h',B',l) \eqdef (s,h \lapp h',B \lapp B',l),\] where $B \lapp B'$ is shorthand for $\lambda i . B_i \lapp B'_i$. 

We must also strengthen the definedness conditions for $\sigma \fseq \sigma'$ from that of $\sigma \fhash \sigma'$ to reflect the desire that all writes of $\sigma$ precede those of $\sigma'$. With the given definition and the definedness conditions from weak interleaving separation, the only scenario in which this condition can be violated is in the case of already present committed writes (in $h'$) conflicting with (i.e., preceding) additional buffered writes (in $B$). In the definition above, all the writes of $h \lapp h'$ implicitly precede all the writes of $B \lapp B'$, and so in particular the writes of $h'$ implicitly precede those of $B$; this is just what we rule out with the following additional definedness condition: \[ B = \nil \disj h' = \lnil \] This strong condition ensures that if any writes have committed in $\sigma'$, then $\sigma \fseq \sigma'$ is only defined if all writes have committed in $\sigma$. 

Observe that if $\sigma \fseq \sigma'$ is defined then $\sigma \fhash \sigma'$ is defined and is refined by the former state (because the concatenation of pomsets is a refinement of the TSO union of pomsets). 

\subsubsection{Strong Sequential Separation}

The strong sequential separation of states, written $\sigma \fsseq \sigma'$ combines the ideas behind the previous two notions of separation: it requires disjointness of the additional memory addresses, and also requires that all additional writes precede those already present. When defined, its value is the same as for weak sequential separation: \[ (s,h,B,l) \fsseq (s,h',B',l) \eqdef (s,h \lapp h', B \lapp B',l).\] The definedness conditions are as follows: \begin{align*}
	\left(\dom{h} \cup \bigcup_{i \in \setprocessors} \dom{B_i} \right) \cap \left( \dom{h'} \cup \bigcup_{i \in \setprocessors} \dom{B'_i}\right) & = \nil \\ 
	B = \nil \disj h' & = \lnil.
\end{align*}

Observe that if $\sigma \fsseq \sigma'$ then $\sigma \fseq \sigma'$ is defined and identical, $\sigma \fsep \sigma'$ is defined and refined by the former state, and $\sigma \fhash \sigma'$ is also defined and refined by the former state.

\subsection{Satisfaction}

The meaning of assertions is given by a satisfaction relation $\sigma \sentails P$, relating states $\sigma$ to assertions $P$. The set of states that satisfies an assertion will be a predicate, as described in Section~\ref{sec:predicates}. 

The satisfaction relation is defined by recursion on the structure of $P$ below in Figure~\ref{fig:satisfaction-relation}. 

\begin{figure}[ht]
	\centering
	\resizebox{\columnwidth}{!}{
	\begin{minipage}{\columnwidth}
	\[
	\begin{array}{llllll}
		(s,h,B,k) & \sentails & {b} & \iffdef & \ext{s}(b) = 1 \\
		(s,h,B,k) & \sentails & {P \disj Q} & \iffdef & (s,h,B,k) \sentails{P} \disj (s,h,B,k) \sentails{Q} \\
		(s,h,B,k) & \sentails & {P \conj Q} & \iffdef & (s,h,B,k) \sentails{P} \conj (s,h,B,k) \sentails{Q} \\
		(s,h,B,k) & \sentails & {\exists x \st P} & \iffdef & \exists v \in \setvalues \st (\funup{s}{\ptup{x}{v}},h,B,k) \sentails{P} \\
		(s,h,B,k) & \sentails & {\forall x \st P} & \iffdef & \forall v \in \setvalues \st (\funup{s}{\ptup{x}{v}},h,B,k) \sentails{P} \\
		(s,\nil,B,\nil) & \sentails & {\femp} & \iffdef & B = (\lambda x . \lnil) \\ 
		(s,h,B,k) & \sentails & {\fbar{e}} & \iffdef & B_{\ext{s}(e)} = \lnil \\ 
		(s,\nil,B,k) & \sentails & {\flock{e}} & \iffdef & B = (\lambda x . \lnil) \conj k = \setprocessors \setminus \set{\ext{s}(e)}\\ 
		(s,h,B,k) & \sentails & {e \fwrite{e'} e''} & \iffdef & k \subseteq \set{\ext{s}(e)} \conj (\forall i \neq \ext{s}(e) \st B_i = \lnil) \conj \\ & & & & ((h = \nil \conj B_{\ext{s}(e')} = \lsingle{(\ext{s}(e),\ext{s}(e''))}) \disj \\ 
		& & & & \,\;(h = \ext{s}(e) \mapsto \ext{s}(e'') \conj B_{\ext{s}(e')} = \lnil \conj k = \nil)) \\
		(s,h,B,k) & \sentails & {P \fhash P'} & \iffdef & \exists \sigma,\sigma' \st \sigma \fhash \sigma' = (s,h,B,k) \conj \\ & & & & \;\;\sigma \sentails{P} \conj \sigma' \sentails{P'} \\
		(s,h,B,k) & \sentails & {P \fsep P'} & \iffdef & \exists \sigma,\sigma' \st \sigma \fsep \sigma' = (s,h,B,k) \conj \\ & & & & \;\;\sigma \sentails{P} \conj \sigma' \sentails{P'} \\
		(s,h,B,k) & \sentails & {P \fseq P'} & \iffdef & \exists \sigma,\sigma' \st \sigma \fseq \sigma' = (s,h,B,k) \conj \\ & & & & \;\;\sigma \sentails{P} \conj \sigma' \sentails{P'} \\
		(s,h,B,k) & \sentails & {P \fsseq P'} & \iffdef & \exists \sigma,\sigma' \st \sigma \fsseq \sigma' = (s,h,B,k) \conj \\ & & & & \;\;\sigma \sentails{P} \conj \sigma' \sentails{P'} \\
		%(s,h,B,k) & \sentails & {\fiter{P}} & \iffdef & (s,h,B,k) \sentails{\femp} \disj (\exists \sigma,\sigma \st \\ & & & & \sigma \sentails P \conj \sigma' \sentails \fiter{P} \conj \sigma \fseq \sigma' = (s,h,B,k))
	\end{array}
	\]
\end{minipage}}
	\caption{\label{fig:satisfaction-relation} The satisfaction relation}
\end{figure}

We write $\pred{P}$ for the set of states that satisfies $P$,\[ \pred{P} \eqdef \setof{\sigma}{\sigma \sentails P},\] and also $P \sentails P'$ and $P \sequiv P'$ for semantic entailment and equivalence, respectively: \begin{align*}
	P \sentails P' \iffdef \pred{P} \subseteq \pred{P'} \\
	P \sequiv P' \iffdef \pred{P} = \pred{P'}.
\end{align*} 

% \begin{proof}
% 	By definition of the satisfaction relation, $\sigma \refines \sigma_1 \pjoin \sigma_2$ for some $\sigma_1,\sigma_2 \in \presat{P}$, and $\sigma' \refines \sigma'_1 \pjoin \sigma'_2$ for some $\sigma'_1,\sigma'_2 \in \presat{P'}$. By Lemma~\ref{foo}, $\sigma \bullet \sigma' \refines (\sigma_1 \bullet \sigma'_1) \pjoin (\sigma_2 \bullet \sigma'_2)$. 
% \end{proof}

\paragraph{Flushing Closure}A central claim is that, for each assertion $P$, $\pred{P}$ is a predicate; i.e., is closed w.r.t.~flushing. An effect of this is that \emph{assertions are oblivious to the nondeterministic flushing of buffered writes to memory}. Intuitively, assertions may intuitively be thought to describe only the ``initial'' states, in which no nondeterministic flushing of writes has taken place, though the semantics encompasses all states reachable as a result these steps. We consider this property to be an important feature of the assertion language---and, hence, of the specification language. 

\begin{lemma}[Flushing closure]
	\label{lem:flushing-closure}
	If $\sigma \sentails P$ and $\sigma' \taurefines \sigma$ then $\sigma' \sentails P$. 
\end{lemma}

Consider, as a significant example, the set of states that satisfy the atomic formula $e \fwrite{e'} e''$. These states may be classified as follows: \begin{enumerate}
	\item states that describe a single buffered write, and 
	\item states in which that write has been committed to memory. 
\end{enumerate} Note that the lock value is unrestricted for the former states, but in the latter the lock is restricted to those processor values for which the buffer is considered live---because a write buffered on a live processor may commit nondeterministically, but a write buffered on a non-live processor may not. 

Consider as another example the states that satisfy the compound assertion $1 \fwrite{2} 3 \fseq 1 \fwrite{2} 4$. The intuitive ``initial'' state is one with two successive writes to location $1$ buffered on processor $2$. Restricting our attention to just those states in which the lock value is $\bot$, the states of the earlier left-side write assertion include ones in which the buffered write has and has not committed, and similarly for the later right-side write assertion. When these two classes states are combined with the sequential separation, three classes of states remain: those in which neither write has flushed, those in which only the earlier write has flushed, and those in which both have flushed. Crucially, the definition of weak sequential separation rules out the case in which the later write has flushed but not the earlier write. This is summarized in Figure~\ref{fig:leads-to-seq-example}. Note that this results in a set of states that is closed w.r.t.~the flushing relation. 
 
\begin{figure}[ht]
	\centering
	\resizebox{\columnwidth}{!}{
	\begin{minipage}{\columnwidth}
	\begin{tabular}{c|c|c}
		$1 \fwrite{2} 3$ &  $1 \fwrite{2} 4$ & $(1 \fwrite{2} 3) \fseq (1 \fwrite{2} 4)$\\ \hline
		$(s,\lnil,2 \mapsto \lsingle{(1,3)},\bot)$ & $(s,\lnil,2 \mapsto \lsingle{(1,4)},\bot)$ & $(s,\lnil,2 \mapsto \lsingle{(1,3)} \lapp \lsingle{(1,4)},\bot)$\\ 
		$(s,\lsingle{(1,3)},\nil,\bot)$ & $(s,\lnil,2 \mapsto \lsingle{(1,4)},\bot)$ & $(s,\lsingle{(1,3)},2 \mapsto \lsingle{(1,4)},\bot)$\\ 
		$(s,\lnil,2 \mapsto \lsingle{(1,3)},\bot)$ & $(s,\lsingle{(1,4)},\nil,\bot)$ & -- \\
		$(s,\lsingle{(1,3)},\nil,\bot)$ & $(s,\lsingle{(1,4)},\nil,\bot)$ & $(s,\lsingle{(1,3)} \lapp \lsingle{(1,4)},\nil,\bot)$\\ 
	\end{tabular}
	\end{minipage}}
	\caption{\label{fig:leads-to-seq-example}Assertion semantics example}
\end{figure}

The requirement that assertions denote sets of states that are closed w.r.t.~flushing also explains why we have chosen atomic formulas that describe empty buffers, $\fbar{e}$, as well as empty heaps and empty buffers, $\femp$. An alternative might be to use one atomic formula to describe empty heaps, say with arbitrary buffers, say $\mathbf{heapemp}$, and another to describe empty buffers with arbitrary heaps. (Then the assertion $\femp$ could be defined as a simple conjunction.) But $\mathbf{heapemp}$ is unsuitable because it does not describe a set of states that is closed under flushing. For if any write is flushed from a buffer in a state with an empty heap, the resulting state would have a heap that is nonempty. 

Also note that it is important to define the presat relation so that it preserves the flushing closure of atomic formulas, as opposed to closing the entire relation at once as with the refinement and partial-join closures.\footnote{Those closures could also be built into the presat relation, I think, but it seemed more concise to factor them out.} E.g., if $\presat{e \fwrite{i} e'}$ is not closed w.r.t.~flushing, then $\presat{\fbar{i} \conj e \fwrite{i} e'}$ is empty, and hence so too would be its flushing closure.  

\subsection{Algebra}
\label{sec:algebra}

A few additional semantic equivalences and entailments are shown in Figures~\ref{fig:equivalences} and~\ref{fig:entailments}, respectively. If a formula contains instances of $\bullet$, then that is short-hand for the same formula in which the $\bullet$ has been consistently replaced by any of the four separating conjunctions. 

\begin{figure}[ht]
	\centering
	\resizebox{\columnwidth}{!}{
	\begin{minipage}{\columnwidth}
	\begin{align*}
		P  \bullet \femp \sequiv & P \\
		\femp \bullet P \sequiv & P \\
		(P \bullet P') \bullet P'' \sequiv & P \bullet (P' \bullet P'') \\
		P \fsep P' \sequiv & P' \fsep P \\
		P \bullet \flock{e} \sequiv & \flock{e} \bullet P \\
		e \fwrite{x} e' \conj \fbar{x} \sequiv & e \fwrite{y} e' \conj \fbar{y}
 	\end{align*}
	\end{minipage}}
	\caption{\label{fig:equivalences}Semantic equivalences}
\end{figure}

We note that the four separating conjunctions naturally form a sort of lattice, and that they satisfy the small exchange laws. The full exchange law only holds for the strong interleaving and sequential conjunctions. The law does not hold for the weak instantiations because, e.g., $P \fhash P'$ is not commutative. 

\begin{figure}[ht]
	\centering
	\resizebox{\columnwidth}{!}{
	\begin{minipage}{\columnwidth}
	\begin{align*}
		P \fsseq P' \sentails & P \fsep P' \\ 
		P \fsseq P' \sentails & P \fseq P' \\ 
		P \fsep P' \sentails & P \fhash P' \\ 
		P \fseq P' \sentails & P \fhash P' \\ 
		P \bullet (P' \circ P'') \sentails & (P \bullet P') \circ P'' & \text{for $P \bullet P' \sentails P \circ P'$} \\
		(P \circ P') \bullet P'' \sentails & P \circ (P' \bullet P'') & \text{for $P \bullet P' \sentails P \circ P'$} \\
		(P \fsep P') \fsseq (P'' \fsep P''') \sentails & (P \fsseq P'') \fsep (P' \fsseq P''') \\
		P \bullet P' \sentails & P'' \bullet P' & \text{if $P \sentails P''$} \\
		P \bullet P' \sentails & P \bullet P'' & \text{if $P' \sentails P''$}
 	\end{align*}
 	\end{minipage}}
	\caption{\label{fig:entailments}Semantic entailments}
\end{figure}


\section{Specifications}
\label{sec:specifications}

The language of specifications is given by the following schema: \[ \spec{J}{P}{c}{Q}, \] where $c$ is a command and $J,P,Q$ are assertions, referred to as the \emph{invariant}, \emph{precondition} and \emph{postcondition}, respectively. 

\subsection{Semantics}
\label{specification-semantics}

A specification asserts the partial correctness of a command. Their informal meaning is roughly analogous to that of concurrent separation logic: if $c$ is evaluated in a state that satisfies $J \fsep P$, then: \emph{1)} it does not abort, \emph{2)} it maintains the invariant $J$ during execution, and \emph{3)} if it evaluates fully, it terminates in a state that satisfies $J \fsep Q$. 

Following Vafeiadis \cite{V11}, the formal semantics of specifications is given by a family of predicates, $\safe{n}{c,\sigma,J,Q}$, parametrized by $n \in \setnaturals$, that relate a command $c$, state $\sigma$, invariant assertion $J$ and postcondition $Q$ according to the informal explanation above. Once these predicates are defined, we define truth of specifications as follows: \[ \truespec{J}{P}{c}{Q} \iffdef \forall \sigma \sentails P \st \forall n \in \setnaturals \st \safe{n}{c,\sigma,J,Q}.\]

In the sequel, let $\locked{\sigma}$ indicate that some processor holds the lock state $\sigma$: \[ \locked{s,h,B,k} \iffdef \exists i \in \setprocessors \st k = \setprocessors \setminus \set{i}. \]

We now give a formal definition of $\safe{n}{c,\sigma,J,Q}$ by natural number induction on $n$. $\safe{0}{c,\sigma,J,Q}$ holds always. And for $n \in \setnaturals$, $\safe{n+1}{c,\sigma,J,Q}$ holds iff the following conditions are true: \begin{enumerate}
	\item If $c = \cskip$ then $\sigma \in Q$.

	\item For all $\sigma_0,\sigma_J,\sigma_F$ such that \begin{enumerate}[(i)]
		\item $\sigma_0 \in (\sigma_J \fsep (\sigma_F \fhash \sigma)$,
		\item $\complete{\sigma_0}$, and
		\item either $\sigma_J \sentails J$ or $\locked{\sigma_1}$,
	\end{enumerate} $c,\sigma_1 \nrightarrow \top$.

	\item For all $\sigma_0,\sigma_J,\sigma_F,\sigma_1,c'$ such that \begin{enumerate}[(i)]
		\item $\sigma_0 \in (\sigma_J \fsep (\sigma_F \fhash \sigma))$,
		\item $\complete{\sigma_0}$,
		\item either $\sigma_J \sentails J$ or $\locked{\sigma_0}$, and 
		\item $c,\sigma_0 \rightarrow c',\sigma_1$,
	\end{enumerate} there exists $\sigma'_J,\sigma'_F,\sigma'$ such that \begin{enumerate}
		\item $\sigma_1 \in (\sigma'_J \fsep (\sigma'_F \fhash \sigma'))$,
		\item $\sigma'_F \taurefines \sigma_F$,
		\item either $\sigma'_J \sentails J$ or $\locked{\sigma_1}$, and
		\item $\safe{n}{c',\sigma',J,Q}$.
	\end{enumerate}

\end{enumerate}

The definition of the predicate above differs from Vafeiadis' in three ways. First, the separating conjunctions are obviously different and, in particular, there are two different separating conjunctions used: one for framing and one for partitioning the local from the shared state. It may be possible (or even necessary) to make use of $\fhash$ uniformly in the definition, which would yield a stronger notion of specification.\footnote{Checking this specifically is next on my agenda. I have in the past proved the soundness of the frame rules with semantics similar to, but not identical to, the one presented.} Second, the frame state is allowed to change from one step to another, but only by making silent transitions. Third, because there is no inherent notion of atomicity, we cannot require that the system state always be separable so that one part satisfies the invariant assertion. For even while one processor holds the lock, others may continue to execute. Hence, we weaken the invariant condition to require only that it hold while the lock is available.

\subsection{Proof Theory}

The axioms of the logic are given in Figure~\ref{fig:axioms}.

\begin{figure}[ht]
	\centering
	\resizebox{\columnwidth}{!}{
	\begin{minipage}{\columnwidth}
		\infax[skip]{\spec{J}{P}{\cskip_i}{P}} 
		\vspace{1em}
		
		\infax[assume]{\spec{J}{!b \disj P}{\cassume{b}_i}{P}}
		\vspace{1em}
		
		\infax[assert]{\spec{J}{b \conj P}{\cassert{b}_i}{P}}
		\vspace{1em}
		
		\infax[assign]{\spec{J}{P\subst{e}{x}}{\cassign{x}{e}_i}{P}}
		\vspace{1em}
		
		\infax[load-buf]{\spec{J}{(e \fwrite{i} e') \fsseq P}{\cload{x}{e}_i}{((e \fwrite{i} e') \fsseq P)\conj x = e'}}
		\vspace{1em}
		
		\infax[load-mem]{\spec{J}{(e \fpointsto e') \fsseq P}{\cload{x}{e}_i}{((e \fpointsto e') \fsseq P) \conj x = e'}}
		\vspace{1em}
		
		\infax[store]{\spec{J}{(e \fpointsto e'') \fseq P}{\cstore{e}{e'}_i}{(e 
		\fpointsto e'') \fseq P \fseq (e \fwrite{i} e')}}
		\vspace{1em}
		
		\infax[fence]{\spec{J}{P}{\cfence_i}{P \conj \fbar{i}}}
		\vspace{1em}
		
		\infax[lock]{\spec{J}{\femp}{\clock_i}{\flock{i}}}
		\vspace{1em}
		
		\infax[unlock]{\spec{J}{\flock{i}}{\cunlock_i}{\fbar{i}}}
	\end{minipage}}
	\caption{\label{fig:axioms}Axioms}
\end{figure}

The axioms for $\cskip$, $\cassume{b}$, $\cassert{b}$ and $\cassign{x}{e}$ are as usual. The two axioms for $\cload{x}{e}$ reflect the fact that the write to be loaded may reside either in the buffer or in the heap. The schematic variable $P$ may be used to describe buffered writes to locations other than $e$ that are more recent than the write to $e$ being loaded, but not more recent writes to $e$. 

The write in the precondition of the store axiom is a witness to the allocation status of $e$. There are other possible axioms for the store command, e.g.: 
	\infax[store-buf]{\spec{J}{(e \fwrite{i} e'') \fsseq P}{\cstore{e}{e'}_i}{((e \fwrite{i} e'') \fsseq P) \fseq e \fwrite{i} e'}}

	\infax[store-mem]{\spec{J}{(e \fpointsto e'') \fsseq P}{\cstore{e}{e'}_i}{((e \fpointsto e'') \fsseq P) \fseq e \fwrite{i} e'}}
These reflect the two cases distinguished by the load axioms, and may or may not be more useful in practice. The axiom for $\cfence$ simply filters away states in which there are buffered writes. 

The axioms for $\clock$ and $\cunlock$ may be strengthened as follows to reflect the fact those commands are associated with implicit fences: 
	\infax[lock]{\spec{J}{P}{\clock_i}{(P \fhash \flock{i}) \conj \fbar{i}}}

	\infax[unlock]{\spec{J}{P \fhash \flock{i}}{\cunlock_i}{P \conj \fbar{i}}}

The logical and structural inference rules of the logic are given in Figures~\ref{fig:logical-inference-rules} and~\ref{fig:structural-inference-rules}, respectively.

\begin{figure}[ht]
	\centering
	\resizebox{\columnwidth}{!}{
	\begin{minipage}{\columnwidth}

		\infrule[disj]{\spec{J}{P}{c}{Q} \text{~~~} \spec{J}{P'}{c}{Q}}{\spec{J}{P \disj P'}{c}{Q}}
		\vspace{1em}

		\infrule[ex]{\spec{J}{P}{c}{Q} \text{~~~} x \notin \fv{c,P}}{\spec{J}{\exists x \st P}{c}{Q}}
		\vspace{1em}

		\infrule[frame-wi]{\spec{J}{P}{c}{Q} \text{~~~} \mod{c} \cap \fv{R} = \nil}{\spec{J}{R \fhash P}{c}{R \fhash Q}}
		\vspace{1em}

		\infrule[frame-si]{\spec{J}{P}{c}{Q} \text{~~~} \mod{c} \cap \fv{R} = \nil}{\spec{J}{R \fsep P}{c}{R \fsep Q}}
		\vspace{1em}

		\infrule[frame-ws]{\spec{J}{P}{c}{Q} \text{~~~} \mod{c} \cap \fv{R} = \nil}{\spec{J}{R \fseq P}{c}{R \fseq Q}}
		\vspace{1em}

		\infrule[frame-ss]{\spec{J}{P}{c}{Q} \text{~~~} \mod{c} \cap \fv{R} = \nil}{\spec{J}{R \fsseq P}{c}{R \fsseq Q}}
		\vspace{1em}

		\infrule[cons]{P \sentails P' \text{~~~}\spec{J}{P'}{c}{Q'} \text{~~~} Q' \sentails Q}{\spec{J}{P}{c}{Q}}

	\end{minipage}}
	\caption{\label{fig:logical-inference-rules}Logical inference rules}
\end{figure}

The logical rules require little explanation. Observe that there is a single left-side frame rule for each of the four separating conjunctions. The strong interleaving conjunction is commutative, and so with the rule of consequence effectively provides a right-side frame rule as well. 

There has been no attempt to provide a conjunction rule because there has been no attempt as yet to identify a suitable notion of precision~\cite{DBLP:journals/entcs/GotsmanBC11}. 

\begin{figure}[ht]
	\centering
	\resizebox{\columnwidth}{!}{
	\begin{minipage}{\columnwidth}

		\infrule[seq]{\spec{J}{P}{c}{R} \text{~~~} \spec{J}{R}{c'}{Q}}{\spec{J}{P}{\cseq{c}{c'}}{Q}}
		\vspace{1em}

		\infrule[choice]{\spec{J}{P}{c}{Q} \text{~~~} \spec{J}{P}{c'}{Q}}{\spec{J}{P}{\cchoice{c}{c'}}{Q}}
		\vspace{1em}

		\infrule[loop]{\spec{J}{P}{c}{P}}{\spec{J}{P}{\cloop{c}}{P}}
		\vspace{1em}

		\infrule[inv]{\spec{\femp}{J \fsep \flock{i,P}}{c}{J \fsep \flock{i,Q}}}{\spec{J}{\flock{i,P}}{c}{\flock{i,Q}}}
		\vspace{1em}

	 	\infrule[conc]{\spec{J}{P}{c}{Q} \text{~~~}\spec{J}{P'}{c'}{Q'} \text{~~~} \fv{P,c,Q} \cap \mod{c'} = \nil \text{, etc.}}{\spec{J}{P \fsep P'}{\cpar{c}{c'}}{Q \fsep Q'}}


	\end{minipage}}
	\caption{\label{fig:structural-inference-rules}Structural inference rules}
\end{figure}

The structural rules for sequential composition, nondeterministic choice, loops and concurrency are as in Concurrent Separation Logic. In particular, the concurrency rule requires agreement between the two processes on the value shared invariant, and the strong interleaving separating conjunction is used to partition the local states. Note that both this conjunction and the concurrent composition command are commutative. Only the invariant rule differs from Concurrent Separation Logic. There, we require that the lock be held while accessing the shared invariant. 

Using the invariant rule and the axioms for $\clock$ and $\cunlock$, we can derive the following rule for reasoning about ``locked'' commands implemented on x86, like atomic increment or compare-and-swap: \infrule[locked]{\spec{\femp}{J \fsep ((\flock{i} \fhash P) \conj \fbar{i})}{c}{J \fsep (\flock{i} \fhash Q)}}{\spec{J}{P}{\cseq{\clock_i}{\cseq{c}{\cunlock_i}}}{Q \conj \fbar{i}}}

\begin{figure}[ht]
	\centering
	\resizebox{\columnwidth}{!}{
	\begin{minipage}{\columnwidth}

		\infrule[shared-load]{\spec{\femp}{J \fsep P}{\cload{x}{e}_i}{J \fsep Q}}{\spec{J}{P}{\cload{x}{e}_i}{Q}}
		\vspace{1em}

		\infrule[shared-store]{\spec{\femp}{J \fsep P}{\cstore{e}{e'}_i}{J \fsep Q}}{\spec{J}{P}{\cstore{e}{e'}_i}{Q}}

	\end{minipage}}
	\caption{\label{fig:daring-structural-inference-rules}Daring structural inference rules}
\end{figure} 

Figure~\ref{fig:daring-structural-inference-rules} shows two additional ``daring'' inference rules, the soundness of which may well be quite difficult to demonstrate. These rules differ from the invariant rule because they allow reasoning about the behavior of individual load and store instructions in which the value of the lock is unspecified. Intuitively, the shared load rule might be shown to be true because a load may only proceed on a live processor, and so will never access shared state while it is being modified by another process, which holds the lock. On the other hand, the shared store axiom might be shown to be true because although a store may take place while another process is modifying the shared state---and hence while the shared state does not satisfy the stated invariant $J$---the buffered write will not commit until the other process has released the lock and repaired the shared state, restoring the invariant.

The daring rules may be needed to reason about, e.g., x86 spinlock implementations. The spinlock is typically acquired using a compare-and-swap instruction, which in this language is simply a locked if-the-else command. The invariant rule and lock axioms thus should be sufficient for demonstrating correctness of spinlock acquisition. But the spinlock is released by writing to a shared memory address without first acquiring the global lock or fencing. This obviates the invariant rule, but not the shared write rule, and so there is yet hope. 

\subsection{Soundness}

A proof is a tree of specifications, in which the leaves are instances of axiom schemas, and the internal specification nodes are instances of the conclusion of some inference rule, with the children of that node as instances of the hypotheses of the inference rule. We write $\overline{\spec{J}{P}{c}{Q}}$ to indicate that there exists a proof tree with the root labeled by $\spec{J}{P}{c}{Q}$. The soundness property simply asserts that provable specifications are true: 

\begin{theorem}[Soundness]
	$\overline{\spec{J}{P}{c}{Q}}$ only if $\truespec{J}{P}{c}{Q}$. 
\end{theorem}

\begin{proof}
	By induction on the structure of an arbitrary proof tree, using the soundness lemmas in Section~\ref{sec:soundness-proofs}. 
\end{proof}

\section{Unresolved Issues}

\begin{enumerate}
	\item The $\fbar{e}$ assertion is additive. It would be better if it described no writes, and if we could express flushing by writing $P \fhash \fbar{e}$ instead of $P \conj \fbar{e}$. Then the fence axiom could be smaller. We might accomplish this by augmenting partial states with the set of processor identifiers that have flushed.  
\end{enumerate}

\section{Conclusion}

\bibliography{dissertation}
\bibliographystyle{abbrv} 

\appendix

\section{Soundness Proofs}
\label{sec:soundness-proofs}
Below, we write $\safe{n}{c,S,J,Q}$, for a set of state $S$, as shorthand for the universal quantification: \[ \forall \sigma \in S \st \safe{n}{c,\sigma,J,Q}.\]

\begin{lemma}
	\label{lem:skip-safe}
	For all $n \in \setnaturals$, if $\sigma \sentails P$ then $\safe{n}{\cskip,\sigma,J,P}$. 
\end{lemma}

\begin{proof}
	By induction on $n$. The base case is trivial. For the induction step, we show $\safe{n+1}{\cskip,\sigma,J,P}$ under the assumption that $\safe{n}{\cskip,\sigma,J,P}$. 
	\begin{enumerate}
		\item Because $c = \cskip$, we must show that $\sigma \sentails P$. But this is true by hypothesis. 

		\item Let $\sigma_0,\sigma_J,\sigma_F$ such that $\sigma_0 \in \sigma_J \fsep (\sigma_F \fhash \sigma)$, $\complete{\sigma_0}$ and either $\sigma_J \sentails J$ or $\locked{\sigma_1}$. We must show that $\cskip,\sigma_0 \nrightarrow \top$. But the only evaluation step possible from configuration $\cskip,\sigma_0$ is by \textsc{c-tau}, which never aborts. 

		\item Let $\sigma_0,\sigma_J,\sigma_F,c_1,\sigma_1$ such that $\sigma_0 \in \sigma_J \fsep (\sigma_F \fhash \sigma)$, $\complete{\sigma_0}$, either $\sigma_J \sentails J$ or $\locked{\sigma_0}$, and $\cskip,\sigma_0 \rightarrow c_1,\sigma_1$. We must show that $\sigma'_J,\sigma'_F,\sigma'$ such that $\sigma_1 \in \sigma'_J \fsep (\sigma'_F \fhash \sigma')$, $\sigma'_F \taurefines \sigma_F$, either $\sigma'_J \sentails \sigma_J$ or $\locked{\sigma_1}$, and $\safe{n}{c_1,\sigma_1,J,P}$.

		The only evaluation step possible from $\cskip,\sigma_0$ is by \textsc{c-tau}, hence $\sigma_1 \taurefines \sigma_0$. By Lemma~\ref{lem:separation-tau}, there exists $\sigma'_J,\sigma'_F,\sigma'$ such that $\sigma_1 \in \sigma'_J \fsep (\sigma'_F \fhash \sigma')$, $\sigma'_J \taurefines \sigma_J$, $\sigma'_F \taurefines \sigma_F$, $\sigma' \taurefines \sigma$. By Lemma~\ref{lem:flushing-closure}, $\sigma'_J \sentails J$ if $\sigma_J \sentails J$. Similarly, $\sigma' \sentails P$ because $\sigma \sentails P$ and $\sigma' \taurefines \sigma$. Hence, by the inductive hypothesis we have that $\safe{n}{\cskip,\sigma',J,P}$. 
	\end{enumerate}
\end{proof}

\begin{lemma}
	\label{lem:skip-sound}
	$\truespec{J}{P}{\cskip}{P}$. 
\end{lemma}

\begin{proof}
	Immediate from Lemma~\ref{lem:skip-safe}. 
\end{proof}

\begin{lemma}
	\label{lem:load-safe}
	If $\sigma \sentails (e \fwrite{e'} e'') \fsseq P$ and $x \notin \fv{e,e',e'',P}$ then, for all $n \in \setnaturals$, $\safe{n}{\sigma,\cload{x}{e}_{e'},J,\left((e \fwrite{e'} e'') \fsseq P \conj x = e''\right)}$. 
\end{lemma}

% \begin{proof}
% 	By induction on $n$. The base case is trivial. For the induction step, we assume the lemma holds for $n$ and show that it holds for $n+1$. \begin{enumerate}
% 		\item 
% 	\end{enumerate}
% \end{proof}

\begin{lemma}
	\label{lem:weak-interleaving-safe}
	If $\safe{n}{c,\sigma,J,Q}$, $\fv{R} \cap \mod{c} = \nil$, $\defined{\sigma_R \fhash \sigma}$, and $\sigma_R \sentails R$, then $\safe{n}{c, \sigma_R \fhash \sigma,J,R \fhash Q}$. 
\end{lemma}

\begin{proof}
	By induction on $n$. The base case is trivial. For the induction step, we assume the lemma holds for $n$ and show that it holds for $n+1$. That is, we assume, for any $c,\sigma,\sigma_R,J,Q,R$, that whenever $\safe{n}{c,\sigma,J,Q}$, $\fv{R} \cap \mod{c} = \nil$, $\defined{\sigma_R \fhash \sigma}$, and $\sigma_R \sentails R$, it is also the case that $\safe{n}{c,\sigma_R \fhash \sigma,J,R \fhash Q}$ holds. We also assume that $\safe{n+1}{c,\sigma,J,Q}$, $\fv{R} \cap \mod{c} = \nil$, $\defined{\sigma_R \fhash \sigma}$, and $\sigma_R \sentails R$, and show that \\ 
	$\safe{n+1}{c,\sigma_R \fhash \sigma,J,R \fhash Q}$. To show this, we must establish three conditions. 

	\begin{enumerate}
		\item Suppose $c = \cskip$. By assumption, $\sigma_R \sentails R$, $\defined{\sigma_R \fhash \sigma}$, and if $c = \cskip$ then $\sigma \sentails Q$, and so $\sigma \sentails Q$. Hence $\sigma_R \fhash \sigma \sentails R \fhash Q$. 

		\item Let $\sigma_0,\sigma_J,\sigma_F$ such that $\sigma_0 \in \sigma_J \fsep (\sigma_F \fhash (\sigma_R \fhash \sigma))$, and either $\locked{\sigma_0}$ or $\sigma_J \sentails J$. We must show that $\sigma_0 \nrightarrow \top$. But \[ \sigma_J \fsep (\sigma_F \fhash (\sigma_R \fhash \sigma)) = \sigma_J \fsep ((\sigma_F \fhash \sigma_R) \fhash \sigma),\] and so by part 2 of assumption $\safe{n+1}{c,\sigma,J,Q}$, instantiating with $\sigma_F \fhash \sigma_R$, the result holds. 

		\item Let $\sigma_0,\sigma_J,\sigma_F,\sigma_1,c'$ such that $\sigma_0 \in \sigma_J \fsep (\sigma_F \fhash (\sigma_R \fhash \sigma))$, either $\locked{\sigma_0}$ or $\sigma_J \sentails J$, and $c,\sigma_0 \rightarrow c',\sigma_1$. We must exhibit $\sigma'_J,\sigma'_F,\sigma'$ such that: \begin{enumerate}[(a)]
			\item $\sigma_1 \in \sigma'_J \fsep (\sigma'_F \fhash \sigma')$,
			\item $\sigma'_F \taurefines \sigma_F$, 
			\item either $\locked{\sigma_1}$ or $\sigma'_J \sentails J$, and
			\item $\safe{n}{c',\sigma',J,R \fhash Q}$. 
		\end{enumerate}

		We instantiate part 3 of assumption $\safe{n+1}{c,\sigma,J,Q}$ with $\sigma_J$, $\sigma_F \fhash \sigma_R$, $\sigma_1$ and $c'$, which gives us $\sigma'_J, \sigma'_{FR}, \sigma'$ such that: \begin{enumerate}[(a)]
			\item $\sigma_1 \in \sigma'_J \fsep (\sigma'_{FR} \fhash \sigma')$
			\item $\sigma'_{FR} \taurefines \sigma_F \fhash \sigma_R$
			\item either $\locked{\sigma_1}$ or $\sigma'_J \sentails J$, and
			\item $\safe{n}{c',\sigma',J,Q}$.
		\end{enumerate}	

		By Lemma~\ref{lem:separation-refinement} and $\sigma'_{FR} \taurefines \sigma_F \fhash \sigma_R$, there exists $\sigma'_F,\sigma'_R$ such that $\sigma'_F \taurefines \sigma_F$, $\sigma'_R \taurefines \sigma_R$ and $\sigma'_{FR} \refines \sigma'_F \fhash \sigma'_R$. 

		From $\sigma'_R \taurefines \sigma_R$ and $\sigma_R \sentails R$, it follows that $\sigma'_R \sentails R$. 

		Next, we wish to apply the inductive hypothesis to $\safe{n}{c',\sigma',J,Q}$ to show that $\safe{n}{c',\sigma'_R \fhash \sigma',J,R \fhash Q}$. This follows from the observations that $\fv{R} \cap \mod{c'} = \nil$ (because $\mod{c'} \subseteq \mod{c}$) and $\defined{\sigma'_R \fhash \sigma'}$. 

		Returning to our original task, we exhibit $\sigma'_J$, $\sigma'_F$ and $\sigma'_R \fhash \sigma'$. By associativity and monotonicity it follows that \[ \sigma_1 \refines \sigma'_J \fsep (\sigma'_F \fhash (\sigma'_R \fhash \sigma')).\] We already showed that $\sigma'_F \taurefines \sigma_F$. Either $\locked{\sigma'_J}$ or $\sigma'_J \sentails J$ by assumption. Finally $\safe{n}{c',\sigma'_R \fhash \sigma',J,R \fhash Q}$ by the inductive hypothesis above. 

	\end{enumerate}
\end{proof}

\begin{lemma}[Weak Interleaving Frame Soundness]
	If $\truespec{J}{P}{c}{Q}$ and $\fv{R} \cap \mod{c} = \nil$ then $\truespec{J}{R \fhash P}{c}{R \fhash Q}$. 
\end{lemma}

\begin{proof}
	Let $\sigma \sentails R \fhash P$ and $n \in \setnaturals$. We must show $\safe{n}{c,\sigma,J,R\fhash Q}$. From $\sigma \sentails R \fhash P$, there exists $\sigma_R \sentails R$ and $\sigma_P \sentails P$ with $\defined{\sigma_R \fhash \sigma_P}$. By assumption, $\safe{n}{c,\sigma_P,J,Q}$. The result follows from Lemma~\ref{lem:weak-interleaving-safe}. 
\end{proof}

\end{document} 
